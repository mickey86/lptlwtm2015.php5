% Copyright 2004 by Till Tantau <tantau@users.sourceforge.net>.
%
% In principle, this file can be redistributed and/or modified under
% the terms of the GNU Public License, version 2.
%
% However, this file is supposed to be a template to be modified
% for your own needs. For this reason, if you use this file as a
% template and not specifically distribute it as part of a another
% package/program, I grant the extra permission to freely copy and
% modify this file as you see fit and even to delete this copyright
% notice. 

\documentclass{beamer}
% Replace the \documentclass declaration above
% with the following two lines to typeset your 
% lecture notes as a handout:
%\documentclass{article}
%\usepackage{beamerarticle}
\usepackage[utf8]{inputenc}
\usepackage[french]{babel}


% There are many different themes available for Beamer. A comprehensive
% list with examples is given here:
% http://deic.uab.es/~iblanes/beamer_gallery/index_by_theme.html
% You can uncomment the themes below if you would like to use a different
% one:
%\usetheme{AnnArbor}
%\usetheme{Antibes}
%\usetheme{Bergen}
%\usetheme{Berkeley}
%\usetheme{Berlin}
%\usetheme{Boadilla}
%\usetheme{boxes}
%\usetheme{CambridgeUS}
%\usetheme{Copenhagen}
%\usetheme{Darmstadt}
%\usetheme{default}
%\usetheme{Frankfurt}
%\usetheme{Goettingen}
%\usetheme{Hannover}
%\usetheme{Ilmenau}
%\usetheme{JuanLesPins}
%\usetheme{Luebeck}
%\usetheme{Madrid}
%\usetheme{Malmoe}
%\usetheme{Marburg}
\usetheme{Montpellier}
%\usetheme{PaloAlto}
%\usetheme{Pittsburgh}
%\usetheme{Rochester}
%\usetheme{Singapore}
%\usetheme{Szeged}
%\usetheme{Warsaw}

\title{PHP~5}

% A subtitle is optional and this may be deleted
\subtitle{en environnement professionnel}

\author{Mikaël~Cordon}

\date{Université de Poitiers, LP TLWTM, 2014-2015}

\subject{Université Poitiers, LP TLWTM, 2014-2015 -- PHP 5}

% \pgfdeclareimage[height=0.5cm]{university-logo}{university-logo-filename}
% \logo{\pgfuseimage{university-logo}}

\AtBeginSection[]
{
  \begin{frame}<beamer>{Vous êtes ici !}
    \tableofcontents[sectionstyle=show/shaded,subsectionstyle=show/hide/hide,subsubsectionstyle=hide]
  \end{frame}
}
\AtBeginSubsection[]
{
  \begin{frame}<beamer>{Vous êtes ici !}
    \tableofcontents[sectionstyle=show/hide,subsectionstyle=show/shaded/hide,subsubsectionstyle=show/hide/hide/hide]
  \end{frame}
}

\begin{document}

\begin{frame}
  \titlepage
\end{frame}

\begin{frame}{Sommaire}
  \tableofcontents[subsectionstyle=hide,subsubsectionstyle=hide]
\end{frame}

\section{Présentation rapide}

\begin{frame}{Historique}
à compléter
    \begin{itemize}
    \item PHP~4 : version très utilisée (autrefois), \textbf{n’est pas} orientée objet
    \item PHP~5 : version de PHP \textbf{orientée objet}
    \item version actuelle (octobre 2014) : 5.6.0
    \end{itemize}

\end{frame}

\begin{frame}{Globalement (?)}
    \begin{block}{}
        \textit{PHP is a popular general-purpose scripting language that is especially suited to web development.\\
        Fast, flexible and pragmatic, PHP powers everything from your blog to the most popular websites in the world.}
    \end{block}
    \begin{itemize}
        \item Logiciel libre : licence GPL
        \item langage à script : interprétation à l’exécution
        \item langage de haut niveau : accès fichiers, sessions, caches, protocole HTTP, les codages caractères (UTF-8)\ldots
        \item beaucoup de bibliothèques disponibles : PEAR, PECL (extensions)
    \end{itemize}
    Documentation de référence : \url{http://www.php.net/manual}

\end{frame}

\begin{frame}{Technologies associées}
Rarement utilisé seul, PHP s’associe, via des bibliothèques et/ou nativement, à :
\begin{itemize}
\item httpd : apache, nginx, lighttpd, IIS, \ldots
\item bases de données : MySQL/MariaDB, PostgreSQL, Oracle, DB2, \ldots
\item présentation : html (html-4.0, XHTML, XML, HTML5), css, javascript, templates, \ldots
\item communications : fichiers, flux, http/s, ftp/s, \ldots
\item \ldots
\end{itemize}

\end{frame}

\section{Architecture de PHP}

\begin{frame}{Fonctionnement client/server}

\end{frame}

\begin{frame}{Installation}

\end{frame}

\begin{frame}{Configuration}

\end{frame}

\begin{frame}{Éléments de configuration importants}

\end{frame}

\section{Les éléments du langage}

%\subsection{Éléments du langage}

%\begin{frame}{Les éléments}
%\begin{itemize}
%\item application ou programme : abstrait, regroupant le reste.
%\item espaces de nom : organise le code et évite les collisions
%\item fichiers : scripts, inclusions, définitions, configurations
%\item classes : déjà définies ou à définir
%\item fonctions ou clôtures : déjà définies ou à définir
%\item structures du langage (\texttt{if}, \texttt{while}, \texttt{switch}, \texttt{\{\}}) : déjà définies
%\item instructions : le code
%\item commentaires et documentation : pour le développeur
%\item variables : \emph{références} à des valeurs
%\item types de valeur (classes) : typage \emph{faible}
%\item valeurs : les données
%\item constantes : valeurs définies une seule fois
%\end{itemize}
%\end{frame}

\subsection{Application}

\begin{frame}{Application PHP}

\end{frame}

\subsection{Espaces de nom}

\begin{frame}{Espaces de nom}

\end{frame}

\subsection{Fichiers}

\begin{frame}{Scripts}

\end{frame}

\begin{frame}{Inclusions}

\end{frame}

\begin{frame}{Fichiers de définition de classe}

\end{frame}

\begin{frame}{Fichiers \texttt{.ini}}

\end{frame}

\subsection{Classes}

\begin{frame}{Classes \guillemotleft~normales~\guillemotright}

\end{frame}

\begin{frame}{Méthodes}

\end{frame}

\begin{frame}{Données de classe}

\end{frame}

\begin{frame}{Instanciation, libération}

\end{frame}

\begin{frame}{Visibilité}

\end{frame}

\begin{frame}{Redéfinition et surcharge}

\end{frame}

\begin{frame}{Héritage}

\end{frame}

\begin{frame}{Classes abstraites}

\end{frame}

\begin{frame}{Interface}

\end{frame}

\subsection{Fonctions, clôtures}

\begin{frame}{Fonctions} %function name()

\end{frame}

\begin{frame}{Paramètres}

\end{frame}

\begin{frame}{Fonctions anonymes} %function ()

\end{frame}

\begin{frame}{Clôtures de code} %traits

\end{frame}

\subsection{Contrôle de l’exécution}

\begin{frame}{Exécution conditionnelle} %if

\end{frame}

\begin{frame}{Exécution en boucle} %while

\end{frame}

\begin{frame}{Exécution sélective} %switch

\end{frame}

\begin{frame}{Regroupement en bloc anonyme} %{}

\end{frame}

\begin{frame}{Exceptions} % try

\end{frame}

\subsection{Instructions}

\begin{frame}{Instruction}

\end{frame}

\begin{frame}{Affectation}

\end{frame}

\begin{frame}{Appels}

\end{frame}

\subsection{Commentaires et documentation}

\begin{frame}{Commentaires}

\end{frame}

\begin{frame}{Documentation}

\end{frame}

\subsection{Variables}

\begin{frame}{Variable}

\end{frame}

\begin{frame}{Référence}

\end{frame}

\subsection{Types de valeurs, valeurs et constantes}

\begin{frame}{Types de valeurs}
Les scalaires :
\begin{itemize}
\item Entiers relatifs : \texttt{int}
\item Flottants : \texttt{float}
\item Chaîne de caractère : \texttt{string}
\item Booléen : \texttt{bool}
\end{itemize}
Les non scalaires :
\begin{itemize}
\item Tableaux : \texttt{array}
\item Objet : \texttt{object}
\end{itemize}
Le pseudo-type \texttt{null}.
\end{frame}

\begin{frame}{Entier}

\end{frame}

\begin{frame}{Flottant}

\end{frame}

\begin{frame}{Chaîne de caractères}

\end{frame}

\begin{frame}{Booléen}

\end{frame}

\begin{frame}{Tableau}

\end{frame}

\begin{frame}{Objet}

\end{frame}

\begin{frame}{Sans type}

\end{frame}

\section{Règles d’écriture}

\begin{frame}{Pourquoi ?}

\end{frame}

\begin{frame}{Éléments concernés}

\end{frame}

\subsection{Quelques exemples/suggestions de règles}

\begin{frame}{Concernant les espacements}

\end{frame}

\begin{frame}{Concernant l’indentation et les accolades}

\end{frame}

\begin{frame}{Concernant la nomenclature des éléments}

\end{frame}

\begin{frame}{Commentaires et documentation}

\end{frame}

\begin{frame}{Types de données, bibliothèques, versions}

\end{frame}

\section{Technologies annexes}

\subsection{Sessions}

\subsection{Authentification}

\subsection{Caches}

\subsection{Codage caractères}

\end{document}
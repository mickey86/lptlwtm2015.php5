% Développer avec PHP5 en entreprise, sur 10 cours
% Mikaël Cordon
% 2014-10-06

% Document maître.

\documentclass[8pt]{beamer}
% Replace the \documentclass declaration above
% with the following two lines to typeset your 
% lecture notes as a handout:
%\documentclass{article}
%\usepackage{beamerarticle}
\usepackage[utf8]{inputenc}
\usepackage[french]{babel}
\usepackage{textcomp}
\usepackage{listings,xcolor}
\usepackage{ulem}
%\usepackage{inconsolata}
\usepackage{wasysym}

%\usetheme{Montpellier}
\usetheme{Antibes}

\title{PHP~5}

% A subtitle is optional and this may be deleted
\subtitle{en environnement professionnel}

\author{Mikaël~Cordon}

\date{Université de Poitiers, LP TLWTM, 2014-2015}

\subject{Université Poitiers, LP TLWTM, 2014-2015 -- PHP 5}

% \pgfdeclareimage[height=0.5cm]{university-logo}{university-logo-filename}
% \logo{\pgfuseimage{university-logo}}

\AtBeginSection[]{
	\begin{frame}<beamer>{Vous êtes ici !}
		\tableofcontents[sectionstyle=show/shaded,subsectionstyle=show/hide/hide,subsubsectionstyle=hide]
	\end{frame}
}
\AtBeginSubsection[]{
	\begin{frame}<beamer>{Vous êtes ici !}
		\tableofcontents[sectionstyle=show/hide,subsectionstyle=show/shaded/hide,subsubsectionstyle=show/hide/hide/hide]
	\end{frame}
}

\newcommand\Pitem{%
	\addtocounter{enumi}{-1}%
	\renewcommand\theenumi{\arabic{enumi}'}%
	\item%
	\renewcommand\theenumi{\arabic{enumi}}%
}

\begin{document}

	\definecolor{dkgreen}{rgb}{0,.6,0}
	\definecolor{dkblue}{rgb}{0,0,.6}
	\definecolor{dkyellow}{cmyk}{0,0,.8,.3}

	\lstset{
		inputencoding   = utf8,
		language        = php,
		%  basicstyle      = \small\ttfamily,
		basicstyle      = \footnotesize\ttfamily,
		keywordstyle    = \color{dkblue},
		stringstyle     = \color{red},
		identifierstyle = \color{dkgreen},
		commentstyle    = \color{gray},
		emph            =[1]{php},
		emphstyle       =[1]\color{black},
		emph            =[2]{if,and,or,else},
		emphstyle       =[2]\color{dkyellow},
		tabsize         = 4,
		showspaces      = false
	}


	\begin{frame}
		\titlepage
		\begin{footnotesize}
			un certain nombre de passages de ce cours sont tirés directement de la documentation PHP.net CC-By PHP Documentation Group.
		\end{footnotesize}
	\end{frame}

	\begin{frame}{Sommaire}
		\tableofcontents[subsectionstyle=hide,subsubsectionstyle=hide]
	\end{frame}


	% Présentation de PHP
	% Cours Présentation de PHP

\section{Présentation de PHP}

\subsection{Présentation}

\begin{frame}{Historique}
    \begin{itemize}
	    \item 1994 : Rasmus Lerdorf, \og~Personal Home Page / Forms Interpreter~\fg
    	\item \og~PHP : Hypertext Preprocessor~\fg 
    	\item PHP~4 (2000) : version très utilisée (autrefois), \textbf{n’est pas} orientée objet
    	\item PHP~5 (2004) : version de PHP \textbf{orientée objet}
    	\item version actuelle (2014-08) : 5.6
    \end{itemize}
\end{frame}

\begin{frame}{Qu'est-ce que PHP ?}
    \begin{block}{\url{http://php.net/}}PHP is a popular general-purpose scripting language that is especially suited to web development.\\Fast, flexible and pragmatic, PHP powers everything from your blog to the most popular websites in the world.
    \end{block}
    \begin{itemize}
        \item Logiciel libre : PHP license v3.01 (BSD-style) ; Documentation : CC-By PHP Documentation Group v3
        \item langage à script : interprétation à l’exécution
        \item langage de haut niveau : accès fichiers, sessions, caches, protocole HTTP, les codages caractères (UTF-8)\ldots
        \item beaucoup de bibliothèques disponibles : PEAR, PECL (extensions)
    \end{itemize}
    Documentation de référence : \url{http://www.php.net/manual}
\end{frame}

\begin{frame}{Technologies associées}
	Rarement utilisé seul, PHP s’associe, via des bibliothèques et/ou nativement, à :
	\begin{itemize}
		\item httpd : apache, nginx, lighttpd, IIS, \ldots
		\item bases de données : MySQL/MariaDB, PostgreSQL, Oracle, DB2, \ldots
		\item présentation : html (html-4.0, XHTML, XML, HTML5), css, javascript, templates, \ldots
		\item communications : fichiers, flux, http/s, ftp/s, \ldots
		\item \ldots
	\end{itemize}
\end{frame}

\subsection{Architecture}

\begin{frame}{Fonctionnement WEB : client/serveur}
	\begin{itemize}
		\item[1] client : le navigateur WEB (Firefox, Chromium, Lynx, IE, \ldots)
			\begin{itemize}
				\item effectue des requêtes : URL (requête HTTP + commande GET, POST, PUT, DELETE)
				\item interprète la réponse reçue : HTML, fichier, CSS, javascript, erreurs (404, 500, 501, \ldots) 
			\end{itemize} 
		\item[2] serveur : httpd 
			\begin{itemize}
				\item httpd ouvre, décode éventuellement la session HTTPS
				\item httpd interprète les URL et appelle le script PHP
				\item PHP exécute le script et génère une réponse HTTP : header HTTP + document
			\end{itemize}
		\item protocole de communication : HTTP 1.1 (éventuellement encapsulé dans un tunnel SSL)
		 	\begin{itemize}
		 		\item transporte la requête HTTP (URL + entêtes)
		 		\item transporte la réponse HTTP (entêtes + document)
		 	\end{itemize}
	\end{itemize}
\end{frame}

\begin{frame}{Installation}
	\begin{itemize}
		\item PHP est un exécutable ; les permissions et droits de l'utilisateur exécutant et ceux sur les fichiers et l'exécutable s'appliquent :
			\begin{itemize}
				\item l'utilisateur et l'environnement de httpd en mode WEB
				\item l'utilisateur courant et son environnement en mode CLI
			\end{itemize}
		\item PHP est livré avec des bibliothèques et des fichiers de configuration
		\item PHP écrit par défaut dans les journaux du service \texttt{httpd} (\texttt{/var/log/apache/error.log} par exemple)
	\end{itemize}
	Installation :
	\begin{itemize}
		\item GNU/Linux : paquet \texttt{php} directement depuis le dépôt (\texttt{.deb}, \texttt{.rpm}, \texttt{.tgz}, \ldots) ; penser aux paquets \texttt{libapachemod-php5}, \ldots
		\item Microsoft Windows : \texttt{WAMP} (Windows Apache MySQL PHP)
		\item Apple MACOSX : \texttt{MAMP} (Mac Apache MySQL PHP), \texttt{AMPPS}, \ldots
	\end{itemize}
\end{frame}

\begin{frame}{Configuration}
	Plusieurs notions :
	\begin{itemize}
		\item environnement : prod, dev, test, \ldots
		\item fonctionnalités : dés/activer des extensions, type et niveau d'erreurs, \ldots
		\item paramétrage : accès mail, accès bdd, \ldots
		\item sécurité : exposition de données sensibles, niveaux de chiffrages, hachage, \ldots
	\end{itemize}
	Plusieurs niveaux :
	\begin{itemize}
		\item global : \texttt{php.ini} (\texttt{/etc/php/})
		\item utilisateur : \texttt{.user.ini} (\texttt{\textasciitilde/}, \texttt{\$HOME/})
		\item service : \texttt{php.ini} (\texttt{/etc/apache/conf.d/})
		\item projet : configuration httpd (\texttt{apache.conf}, \texttt{.htaccess})
		\item \emph{on-the-fly} : \texttt{ini\_get()} et \texttt{ini\_set()}
	\end{itemize}
\end{frame}

\begin{frame}{Quelques éléments de configuration}
	Dans \texttt{php.ini} :
	\begin{itemize}
		\item \texttt{max\_execution\_time=30}, \texttt{memory\_limit=128M} : limiter les ressources utilisées par chaque instance de script
		\item \texttt{post\_max\_size=8M}, \texttt{upload\_max\_filesize=2M} : limiter la taille des données
		\item \texttt{session.use\_cookies=1} : paramètres d'extension (ici, \texttt{session})
		\item \ldots
	\end{itemize}
\end{frame}

	% Exercice Hello World


\section{\texttt{Hello world!}}

\begin{frame}[containsverbatim]{Hello world!}
Dans un fichier texte, avec votre éditeur préféré\ldots 
\begin{block}{index.html}
\begin{lstlisting}
<!DOCTYPE html>
<html>
    <head>
        <title>PHP Test</title>
    </head>
    <body>
        <?php echo '<p>Hello World</p>' ; ?>
    </body>
</html>
\end{lstlisting}
\end{block}
Placez-le dans \texttt{/var/www/html}.\\
Visitez \url{http://localhost/index.html} et analysez la requête et la réponse HTTP.
\end{frame}


	% Description du langage
	% Cours Les éléments du langage

\section{Les éléments du langage}

\subsection{Types de valeurs, valeurs et constantes}

\begin{frame}{Types de valeurs}
	Les scalaires :
	\begin{itemize}
		\item Entiers relatifs : \texttt{int}
		\item Flottants : \texttt{float}
		\item Chaîne de caractères : \texttt{string}
		\item Booléen : \texttt{bool}
	\end{itemize}
	Les non scalaires :
	\begin{itemize}
		\item Tableau : \texttt{array}
		\item Objet : \texttt{object}
		\item Ressource : \texttt{resource}
	\end{itemize}
	Le pseudo-type \texttt{null}
\end{frame}

\begin{frame}[containsverbatim]{Entier}
	\begin{itemize}
		\item entier relatif : $[-2^{31}$ ; $2^{31} - 1]$ ou  $[-2^{63}$ ; $2^{63} - 1]$ (32 ou 64 bits selon l'architecture système)
		\item est pris automatiquement pris pour un entier si seul dans une chaîne de caractères
		\item peut être représenté dans différentes bases (2, 8, 10, 16) 
	\end{itemize}
	\begin{block}{code : entiers}
		\begin{lstlisting}
<?php
	$a = 1234 ; // un nombre decimal
	$a = -123 ; // un nombre negatif
	$a = 0123 ; // un nombre octal (equivalent a 83 en decimal)
	$a = 0x1A ; // un nombre hexadecimal (equivalent a 26 en decimal)
	$a = 0b11111111 ; // un nombre binaire (equivalent a 255 en decimal)
		\end{lstlisting}
	\end{block}
\end{frame}

\begin{frame}[containsverbatim]{Flottant}
	\begin{itemize}
		\item nombres à partie décimale (14 chiffres de précision) sur 64 ou 128 bits selon l'architecture système
		\item environ $1,8 \times 10^{308}$ est un maximum fréquent
		\item s'écrit avec la convention anglosaxone (point et pas virgule pour le séparateur décimal)  
	\end{itemize}
	\begin{block}{code : flottants}
		\begin{lstlisting}
<?php
	$a = 1.234 ;
	$b = 1.2e3 ;
	$c = 7E-10 ;
		\end{lstlisting}
	\end{block}
\end{frame}

\begin{frame}[containsverbatim]{Chaîne de caractères}
	\begin{itemize}
		\item suite de caractères, de longueur 0 à 2Gio, codage unicode (UTF-8 ou UTF-16) en interne PHP, mais il faut transcoder explicitement ou automatiquement
		\item utilisation d'échappement des caractères : \texttt{\textbackslash{}n}, \texttt{\textbackslash{}r}, \texttt{\textbackslash{}t}, \texttt{\textbackslash{}v}, \texttt{\textbackslash{}\textbackslash{}}, \texttt{\textbackslash{}'}, \texttt{\textbackslash{}"}, \texttt{\textbackslash{}\$}, \ldots
		\item chaînes de caractères littérales : pas de substitution de variable possible
		\item chaînes de caractères interprétable : des substitutions de variables peuvent intervenir
		\item quatre représentations possibles : \texttt{''}, \texttt{""}, \texttt{nowdoc}, \texttt{heredoc}
	\end{itemize}
	\begin{block}{code : chaînes de caractères}
		\begin{lstlisting}
<?php
	$n = 27 ;
	echo 'une chaine de $n caracteres' ; // une chaine de $n caracteres
	echo "une chaine de $n caracteres" ; // une chaine de 27 caracteres
	
	$chaine = <<<"EOCHAINE"
Une chaine de caracteres NOWDOC sur plusieurs
lignes, de plus de $n caracteres.
EOCHAINE;
/* Une chaine de caracteres NOWDOC sur plusieurs
lignes, de plus de 27 caracteres. */

	$chaine = <<<EOCHAINE
Une chaine de caracteres HEREDOC sur plusieurs
lignes, de plus de $n caracteres.
EOCHAINE;
/* Une chaine de caracteres HEREDOC sur plusieurs
lignes, de plus de $n caracteres. */
		\end{lstlisting}
	\end{block}
\end{frame}

\begin{frame}[containsverbatim]{Booléen}
	\begin{itemize}
		\item \texttt{true} et \texttt{false}
		\item les autres types de valeurs peuvent être interprétés comme des booléens
	\end{itemize}
	\begin{block}{code : booléens}
		\begin{lstlisting}
<?php
	$a = true ;
	$b = false ;
	$c = $a && $b ; // false !
	$c == false ; // true !
	$c == "" ; // true !
	$c === "" ; // false !?
		\end{lstlisting}
	\end{block}
\end{frame}

\begin{frame}[containsverbatim]{Tableau}
	\begin{itemize}
		\item représente une liste ordonnée de valeurs
		\item les valeurs peuvent être de tout type PHP
		\item les valeurs sont indicées par un scalaire : entier, chaîne de caractère, flottant, \ldots
		\item les valeurs et les indices peuvent être de types différents dans un même tableau, et chaque élément peuvent accueillir successivement des valeurs de types différents
	\end{itemize}
	\begin{block}{code : tableaux}
		\begin{lstlisting}
<?php
	$a = array () ;
	$b = array (1, 2, $a) ;
	$c = array (1 => "toto", 'b' => "tata") ;
	$d = array (
		array (true => "vrai"),
		false => 2,
	) ;
	$e = [1, &$b, 3, "a", ["g" => 6, new MaClasse ()]] ;
	$e[4] = "b" ;
	$e[5]['g'] = null ;
	$e = [1, &$b, 3, "b", array ('g' => null, new MaClasse ())] ; // True !
		\end{lstlisting}
	\end{block}
\end{frame}

\begin{frame}[containsverbatim]{Objet}
	\begin{itemize}
		\item découle directement de l'instanciation d'une classe
		\item l'accès aux membres de la classe se fait via l'opérateur \texttt{->}
		\item sa valeur découle directement de ses propriétés 
	\end{itemize}
	\begin{block}{code : objets}
		\begin{lstlisting}
<?php
	class foo {
		function do_foo () {
			echo "Doing foo." ;
		}
	}

	$bar = new foo () ;
	$bar->do_foo () ;
		\end{lstlisting}
	\end{block}
\end{frame}

\begin{frame}[containsverbatim]{Sans type}
	\begin{itemize}
		\item \texttt{null} représente une absence de valeur
		\item lorsqu'on affecte \texttt{null} à une variable, elle est libérée et les références à la valeur remplacée aussi !
	\end{itemize}
	\begin{block}{code : null}
		\begin{lstlisting}
<?php
	$a = null ;
	isset ($a) == false ; // true !
	$b = 1 ;
	$c = & $b ;
	$b = null ;
	$c == null ; // true !
	array () == null ; // false !
	"" == null ; // false !
	false == null ; false !
	$m = new MaClasse () ;
	unset ($m) ;
	$m == null ; // true !
	$n = new MaClasse () ;
	$n = null ;
	$n == null ; // true, evidemment.
		\end{lstlisting}
	\end{block}
\end{frame}

\subsection{Variables}

\begin{frame}[containsverbatim]{Variable}
	\begin{itemize}
		\item représente une valeur nommée ; ou peut ne pas avoir de valeur (\texttt{null})
		\item est, implicitement (en interne), un pointeur sur une valeur
		\item \texttt{\$[\_a-zA-Z][\_a-zA-Z0-9]*}, des variables sont déjà définies par PHP, certaines peuvent être reécrites, d'autres non : \texttt{\$GLOBALS}, \texttt{\$\_SERVER}, \ldots
		\item n'est pas typée, peut contenir successivement plusieurs valeurs de types différents
		\item a un domaine d'existence : portée ou scope
	\end{itemize}
	\begin{block}{code : variables}
		\begin{lstlisting}
<?php
	$a = 1 ;
	$_MaVariable = "Ma Valeur" ;
	$Tableau = array ("toto", 5) ;
	$Objet = new user ('toto') ;
	$texte = $GLOBALS["Objet"]->say () ;
	$var = '_MaVariable' ;
	$test = $$var == array ("toto", 5) ; // True !
	function add ($a, $b) {
		return $a + $b ;
	}
		\end{lstlisting}
	\end{block}
\end{frame}

\begin{frame}[containsverbatim]{Référence}
	\begin{itemize}
		\item une référence peut être vue comme un pointeur, mais \textbf{ne} peut \textbf{pas} être manipulé en tant que tel (ainsi qu'on peut le faire avec des pointeurs en \texttt{C})
		\item toute variable, peut être vue comme un pointeur sur une valeur, certaines variables pointent sur la même instance de valeur (même zone mémoire), on les appelle des références
		\item une valeur n'est plus accessible dès lors que toutes les références à celle-ci sont détruites (via \texttt{unset ()} ou \texttt{= null})
		\item une référence à une valeur est créée avec l'opérateur \texttt{\&}
	\end{itemize}
	\begin{block}{code : références}
		\begin{lstlisting}
<?php
	$a = 1
	$b = & $a ;
	$b = 2 ;
	$a == 2 ; // True !
	unset ($a) ;
	$b == 2 ; // True !
	function & incremente (&$a, $b = 1) {
		$a += $b ;
		return $a ;
	}
		\end{lstlisting}
	\end{block}
\end{frame}

\subsection{Constantes}

\begin{frame}[containsverbatim]{Constantes}
	\begin{itemize}
		\item valeur nommée, mais inchangeable
		\item le nom de la constante suit les mêmes règles de nommage des variables, mais sans le \texttt{\$}
		\item habituellement, en PHP, les noms des constantes sont en majuscule
		\item des constantes magiques existent (des constantes qui changent de valeur :) ) : \texttt{\_\_LINE\_\_}, \texttt{\_\_FILE\_\_}, \texttt{\_\_DIR\_\_}, \texttt{\_\_FUNCTION\_\_}, \texttt{\_\_CLASS\_\_}, \ldots
	\end{itemize}
	\begin{block}{code : constantes}
		\begin{lstlisting}
<?php
	define ("MA_CONSTANTE", "sa valeur") ;
	echo MA_CONSTANTE ;
		\end{lstlisting}
	\end{block}
\end{frame}

\subsection{Commentaires et documentation}

\begin{frame}[containsverbatim]{Commentaires}
	\begin{itemize}
		\item destinés aux développeurs (et aux curieux)
		\item démystifient le code, explicite un code un peu confus ou complexe ou optimisé
		\item permettent la lecture rapide d'un code sans avoir à le décrypter
		\item doivent toujours être à jour vis-à-vis du code ! 
	\end{itemize}
	\begin{block}{code : commentaires}
		\begin{lstlisting}
<?php
	// Si $a est non null et que la valeur incrementee n'est pas zero...
	if (isset ($a) && $a ++) {
		$b += $a ; // ...on ajoute $a a $b
	}
	
	// ------------8<--------------
	// parfois des modifications qu'il faudra enlever avant la prod
	// ------------>8--------------	
	
	/* Un commentaire
	sur plusieurs lignes si necessaire. */
		\end{lstlisting}
	\end{block}
\end{frame}

\begin{frame}[containsverbatim]{Documentation}
	\begin{itemize}
		\item destinée aux développeurs et utilisateurs du code
		\item renseigne sur le comportement et la composition des objets définis (fichiers, classes, méthodes, fonctions, variables, constantes, \ldots)
		\item doit être systématique, exhaustive, à jour, structurée !
		\item doxygen, javadoc, \ldots
	\end{itemize}
	\begin{block}{code : documentation}
		\begin{lstlisting}
<?php
	/** Classe representant les donnees personnelles d'un utilisateur. */
	class user {
	
		/** nom de l'utilisateur : chaine de 32 caracteres UTF-8 */
		private $name ;
		
		/** Constructeur de la classe
		  * Initialise le nom de l'utilisateur.
		  * \param $name : une chaine de caracteres, tronquee a 32 caracteres, 'toto' si non renseigne ou null.
		  */
		function __construct ($name = 'toto') {
			$this->name = substr (0, 32, $name) ;
		}
		
		/** L'utilisateur se presente.
		  * \return une chaine de caractere presentant l'utilisateur.
		  */
		function say () {
			return "Bonjour, je m'appelle {$this->name}." ;
		}
	}
		\end{lstlisting}
	\end{block}
\end{frame}

\subsection{Expressions}

\begin{frame}[containsverbatim]{Expression}
	Combinaison (via des opérateurs ou des fonctions) de valeurs qui retourne une valeur
	\begin{itemize}
		\item \lstinline!1!, \lstinline!array ('nom' => 'toto')!, \lstinline!$t['nom']! : une valeur ; \lstinline!$a! : une valeur de variable
		\item \lstinline!$a + sqrt (2)!, \lstinline!$a <= 5!, \lstinline!$a ++! : une expression mathématique, booléenne
		\item \lstinline!strlen ('toto' . "titi $a")! : un appel de fonction
		\item \lstinline!isset ($a) ? "" : 3! : une opération quelconque
		\item \lstinline!(1 + strlen ('toto')) * 2)! : gestion des priorités avec \texttt{()}
		\item \lstinline!$a = 1! : est une expression, elle retourne la valeur de \lstinline!$a! après affectation
		\item n'importe quelle combinaison d'expressions
		\item le type résultant de l'expression est déterminé lors de son évaluation (typage faible)
	\end{itemize}
	Ne sont pas considérées comme expressions :
	\begin{itemize}
		\item \lstinline!if ($a == null) { /* ... */ }!, \lstinline!while (true) { /* ... */ }!
		\item \lstinline*echo "hello world!"* : \texttt{echo} n'est pas une fonction, mais une structure du langage PHP
		
	\end{itemize}
\end{frame}

\begin{frame}[containsverbatim]{Opérateurs}
	\begin{itemize}
		\item La précédence des opérateurs : \texttt{()} (\url{http://fr2.php.net/manual/en/language.operators.precedence.php})
		\item Les opérateurs arithmétiques : \lstinline!-$a!, \lstinline!$a + $b!, \lstinline!$a - $b!, \lstinline!$a * $b!, \lstinline!$a / $b!, \lstinline!$a % $b!, \lstinline!$a ** $b! 
		\item Les opérateurs d'affectation : \lstinline!$a = 0!, \lstinline!$a += 1!, \lstinline!$a -= 1!, \lstinline!$a *= 1!, \lstinline!$a /= 1!, \lstinline!$a .= 'toto'!, \ldots
		\item Opérateurs sur les bits : \lstinline!~$a!, \lstinline!$a & $b!, \lstinline!$a | $b!, \lstinline!$a ^ $b!, \lstinline!$a << $b!, \lstinline!$a >> $b!
		\item Opérateurs de comparaison : \lstinline!$a == $b!, \lstinline!$a === $b!, \lstinline*$a != $b*, \lstinline!$a <> $b!, \lstinline!$a < $b!, \lstinline!$a > $b!, \lstinline!$a <= $b!, \lstinline!$a >= $b!, \lstinline*$a !== $b*, 
		\item Opérateur de contrôle d'erreur : \lstinline!@ $a / 0! (ignore les erreurs générées par l'expression, bof :/)
		\item Opérateur d'exécution : \lstinline!`ls -lA`! (exécute une commande système et retourne la sortie standard)
		\item Opérateurs d'incrémentation et décrémentation : \lstinline!$a ++!, \lstinline!$a --!, \lstinline!-- $a!, \lstinline!++ $a!
		\item Les opérateurs logiques : \lstinline!$a and $b!, \lstinline!$a or $b!, \lstinline!$a xor $b!, \lstinline*!$a*, \lstinline!$a && $b!, \lstinline!$a || $b! 
		\item Opérateurs de chaînes : \lstinline!$a . $b!, \lstinline!$a .= $b!
		\item Opérateurs de tableaux : \lstinline!$a + $b!, \lstinline!$a == $b!, \lstinline!$a === $b!, \lstinline*$a != $b*, \lstinline!$a <> $b!, \lstinline*$a !== $b*
		\item Opérateurs de types : \lstinline!$a instanceof 'Classe'!, \lstinline!$a instanceof Classe!
	\end{itemize}
\end{frame}

\subsection{Instructions}

\begin{frame}[containsverbatim]{Instruction}
	\begin{itemize}
		\item Affectation ou expression
		\item finit par \texttt{;} (ou \texttt{?>})
		\item en général, une par ligne
		\item une instruction retourne une valeur (la plupart du temps)
	\end{itemize}
	\begin{block}{code : instructions}
		\begin{lstlisting}
<?php
	$a = 1 ;
	$a ++ ;
	$c = new MaClasse () ;
	printf ("%s", $c) ;
	return 'toto' ;
	'toto' ; // pas tres utile
	$a = $b = (int) 3 + strlen ((string) $c) ;
	$a = $b == $c ;
	echo $a ;
		\end{lstlisting}
	\end{block}
\end{frame}

\subsection{Contrôle de l’exécution}

\begin{frame}[containsverbatim]{Exécution conditionnelle} %if
	\begin{block}{code : if}
		\begin{lstlisting}
<?php
	if (/* expression */) {
		/* instructions */
	}
	
	if (/* expression */) {
		/* instructions */
	} else {
		/* instructions */
	}
	
	if (/* expression */) {
		/* instructions */
	} elseif { // ou else if
		/* instructions */
	}
		\end{lstlisting}
	\end{block}
\end{frame}

\begin{frame}[containsverbatim]{Exécution en boucle} %while
	\begin{itemize}
		\item boucle conditionnelle (tant que) : \texttt{while () \{\}}
		\item boucle sur liste ou tableau (pour chaque élément) : \texttt{foreach (as) \{\}}
		\item boucle itérative : \texttt{for (;;) \{\}}
		\item contrôle des boucles via : \lstinline!break ;!, \lstinline!continue ;!
	\end{itemize}
	\begin{block}{code : for, foreach, while}
		\begin{lstlisting}
<?php
	while (/* expression */) {
		/* instructions */
	}
	
	$tableau = array (/* .. */) ;
	foreach ($tableau as $element) {
		/* instruction */
	}
	
	foreach ($tableau as $index => $element) {
		/* instructions */
	}
	
	for ($i = 0 ; $i < 10 ; $i ++) {
		/* instructions */
	}
		\end{lstlisting}
	\end{block}
\end{frame}

\begin{frame}[containsverbatim]{Exécution sélective} %switch
	\begin{itemize}
		\item exécution conditionnelle selon plusieurs valeurs scalaires
	\end{itemize}
	\begin{block}{code}
		\begin{lstlisting}
<?php
	$n = (int) random (10) ;
	switch ($n) {
	case 1 :
		echo 'unite' ;
		break ;
	case 2 :
	case 4 :
		echo 'pair inferieur a 5' ;
		break ;
	case 3 :
		echo 'impair inferieur a 5' ;
		break ;
	default :
		echo 'superieur ou egal a 5' ;
	}
		\end{lstlisting}
	\end{block}
\end{frame}

\begin{frame}[containsverbatim]{Exceptions} % try
	\begin{itemize}
		\item permet de capturer une erreur d'exécution, de traiter l'erreur et finir le boulot
		\item les exceptions retournées par PHP sont des objets de classes dérivant de \texttt{Exception}
	\end{itemize}
	\begin{block}{code}
		\begin{lstlisting}
<?php
	function diviser ($a, $b) {
		try {
			if ($b == 0) { 
				throw new Exception ("Division par 0.") ;
			}
			return $a / $b ;
		} catch (Exception $e) {
			return null ;
		}
	}
		\end{lstlisting}
	\end{block}
\end{frame}

\subsection{Fonctions}

\begin{frame}[containsverbatim]{Fonctions} %function name()
	\begin{itemize}
		\item déclaration : signature + corps (sauf pour les méthodes abstraites)
		\item l'identificateur (nom) : \texttt{[\_a-zA-Z][\_a-zA-Z0-9]*}
		\item requiert zéro ou plusieurs paramètres selon la signature
		\item le corps est compris entre \{ et \}, démarre par la première instruction et finit par la dernière instruction ou par l'instruction \lstinline!return /* ... */ ;!
		\item retourne une seule valeur (via \texttt{return}) ou une référence (défini dans la signature par \&) vers la valeur
		\item a un domaine d'existence : portée, scope
	\end{itemize}
	\begin{block}{déclaration}
		\begin{lstlisting}
<?php
function ajouter ($a, $b = 0) {
	return $a + $b ;
}

function & incrementer (&$a, $b = 1) {
	$a += $b ;
	return $a ;
}
		\end{lstlisting}
	\end{block} 
\end{frame}

\begin{frame}[containsverbatim]{Paramètres de fonctions}
	Ou arguments.
	\begin{itemize}
		\item passage par valeur : \lstinline!function afficher ($a) { echo $a ; }!, une copie de la valeur est effectuée au moment de l'appel de la fonction, la valeur \texttt{\$a} dans la fonction est indépendante de la valeur du code appelant
		\item passage par référence : \lstinline!function incrementer (& $a) { $a ++ ; }!, une référence de la valeur est passée à la fonction, toute modification de \texttt{\$a} dans la fonction est effectuée sur la valeur de la variable dans le code appelant
		\item un paramètre peut être optionnel, mais il lui faut une valeur par défaut : \lstinline!function ajouter ($a, $b = 1) { return $a + $b ; }!, lors de l'appel de cette fonction, on peut omettre le paramètre (\lstinline!$c = ajouter (10, 1) ; $d = ajouter (2) ;!)
		\item un paramètre peut être typé (PHP $\ge$ 5.1) par \texttt{array}, \texttt{objet}, \texttt{callable}, mais pas par les types scalaires !
		\item lors de l'appel, on peut préciser plus de paramètres que déclarés dans la signature : liste variable de paramètres. Dans la fonction, on les récupère en tant que tableau de valeurs (ou de références) via les fonctions \texttt{func\_get\_args ()}, \texttt{func\_num\_args ()}
		\item depuis PHP 5.6, on peut utiliser le mot clé \texttt{...} avant le paramètre dans la signature pour préciser que tous les paramètres suivants sont stockés dans ce paramètre : \lstinline!function ajouter (...$num) { $s = 0 ; foreach ($num as $n) { $s += $n ; } return $s ; }! 
	\end{itemize}
\end{frame}

\begin{frame}[containsverbatim]{Fonctions anonymes} %function ()
	\begin{itemize}
		\item fonction à usage unique ou jetable, une fonction anonyme n'a pas de nom : \lstinline!function ($a) { /* corps */ }!
		\item s'utilise dans certains contextes
	\end{itemize}
	\begin{block}{code : fonctions anonymes}
		\begin{lstlisting}
<?php
$ajouter = function ($a, $b) { return $a + $b ; }
$s = $ajouter (1, 2) ;

$a = array (/* ... */) ;
$b = array_walk (function ($e) { return $e + 1 ; }, $a) ;
		\end{lstlisting}
	\end{block}
\end{frame}


\subsection{Classes}

\begin{frame}[containsverbatim]{Les classes}
	PHP $\ge$ 5.0.0 devient orienté objet.
	\begin{itemize}
		\item notion de classes classique : classe, objet/instance, abstraite, interface, héritage (simple), surcharge, \ldots
		\item \texttt{traits} \ldots
		\item définition et implémentation dans le même fichier (différent de java, par exemple)
	\end{itemize}
	\begin{block}{définition de classe}
		\begin{lstlisting}
// definition
class A {

	private $var ;

	function afficher () { /* ... */ } 
	/* ... */
} 

// heritage
class B extends A { /* ... */ } 

// definition d'interface
interface iC { /* ... */ } 

// definition classe abstraite et heritage d'interface
abstract class D extends iC { /* ... */ } 
		\end{lstlisting}
	\end{block}
\end{frame}

\begin{frame}[containsverbatim]{Les classes : \guillemotleft~normales~\guillemotright}
	\begin{block}{définition de classe}
		\begin{lstlisting}
class A {

	/** Ma donnee : un entier */
	private $var ;

	/** Constructeur
	  * Initialise ma donnee a 0.
	  */
	function __construct () {
		$this->var = 0 ;
	} 
	
	/** Affiche ma donnee avec un suffixe.
	  * \param texte : suffixe
	  */
	public function afficher ($texte) {
		echo $this->var, " ", $texte ;
	}
	
	/** Ajoute 1 a ma donnee.
	  */
	public function incrementer () {
		$this->var ++ ;
	} 
} 
		\end{lstlisting}
	\end{block}
\end{frame}

\begin{frame}[containsverbatim]{Les classes : les méthodes}
	\begin{itemize}
		\item \lstinline!public function nom ($parametre) { /* corps */ }! ou \lstinline!function nom ($parametre) { /* corps */ }! : méthode publique, utilisable par tout le monde, redéfinissable dans les classes descendantes
		\item \lstinline!private function nom ($param1, $param2) { /* corps */}! : méthode privée, utilisable depuis un objet de la même classe
		\item \lstinline!protected function nom ($param1, $param2) { /* corps */}! : méthode protégée, utilisable depuis un objet de la même classe ou descendante, redéfinissable
		\item \lstinline!static function nom ($param) { /* corps */}! : méthode statique ou méthode de classe, utilisable sans instancier la classe
		\item \lstinline!final function nom ($param) { /* corps */}! : méthode finale, non redéfinissable
		\item \lstinline!abstract function nom ($param) ;! : méthode abstraite, sans corps, obligatoirement redéfinie dans la classe descendante, implique que la classe où elle est déclarée est aussi abstraite.
		\item \lstinline!public function __get ($name) { /* corps */ }! : méthode magique (ici un getter, surcharge de l'opération de lecture d'une propriété), \ldots
	\end{itemize}
\end{frame}

\begin{frame}[containsverbatim]{Les classes : les propriétés}
	Attributs, membres, champs, \ldots
	\begin{itemize}
		\item \lstinline!public $var ;! : déclaration d'une propriété publique (accessible par tout le monde), sans initialisation
		\item \lstinline!public $var = 'valeur' ;! : déclaration avec une valeur initiale
		\item \lstinline!private $var = array (10, 'val') ;! : propriété privée, la valeur initiale est un tableau de valeurs
		\item \lstinline!protected $var = 6 ;! : propriété protégée
		\item \lstinline!static $var = 2 ;! : propriété statique, propriété de classe, accessible lorsque la classe n'est pas instanciée
	\end{itemize}
\end{frame}

\begin{frame}[containsverbatim]{Les classes : instanciation, libération}
	Instanciation : 
	\begin{itemize}
		\item \lstinline!$o = new MyClass ('val') ;!
		\item crée une instance de la classe MyClass, appelle la méthode \lstinline!MyClass::__construct ($var)! si elle est définie
	\end{itemize}
	Libération :
	\begin{itemize}
		\item \lstinline!unset $o ;!
		\item libère l'objet \lstinline!$o! après avoir appelé la méthode \lstinline!MyClass::__destruct ()! si elle est définie
		\item ne libère pas automatiquement les propriétés de type objet ou référencés, à faire explicitement dans \lstinline!__destruct ()!
	\end{itemize}
\end{frame}

\begin{frame}[containsverbatim]{Les classes : visibilité}
	Concerne les méthodes et les propriétés, statiques ou pas.
	\begin{itemize}
		\item \texttt{public} : publique, pas de restriction de visibilité
		\item \texttt{protected} : protégée, visibilité et utilisabilité restreinte à la classe, ses descendants et ascendants
		\item \texttt{private} : privée, visibilité et utilisabilité restreinte à la classe où c'est déclaré 
		\item sans mot clé \texttt{public}, \texttt{protected}, \texttt{private}, la visibilité par défaut est publique.
		\item \lstinline!var $var ;! est identique à \lstinline!public $var ;!
		\item Lors de l'héritage, les classes descendantes peuvent restreindre la visibilité, pas l'étendre : \texttt{public} \textrightarrow \texttt{protected}, \texttt{public} \textrightarrow \texttt{private}, \texttt{protected} \textrightarrow \texttt{private}, \sout{\texttt{public} \textleftarrow \texttt{private}}, \ldots
	\end{itemize}
\end{frame}

\begin{frame}[containsverbatim]{Les classes : redéfinition et surcharge}
	Redéfinition de méthode :
	\begin{itemize}
		\item changement de comportement de la méthode de même nom dans la classe fille
		\item on ne redéfinit que les méthodes accessibles depuis la classe fille, la visibilité de la redéfinition est la même ou restreinte
		\item l'ancêtre de la méthode redéfinie reste accessible en la nommant explicitement : \lstinline!ClasseParent::MaMethode () ;! ou \lstinline!parent::MaMethode () ;!
	\end{itemize}
	Surcharge de méthode :
	\begin{itemize}
		\item rajoute du comportement à la méthode de même nom dans la classe fille
		\begin{itemize}
			\item appelle dans le corps de la méthode \lstinline!MaMethode()! la méthode ancêtre via \lstinline!parent::MaMethode()!
			\item se fait notamment pour \lstinline!__construct ()! et \lstinline!__destruct ()!
		\end{itemize}
		\item rajoute un comportement à une opération sur la classe ou l'objet : méthode magique
		\begin{itemize}
			\item \lstinline!__get ($prop)! : est appelée lors de l'accès à la propriété \lstinline!$prop! inaccessible (restreinte ou inexistante)
			\item \lstinline!__set ($prop, $value)! : est appelée lors de l'affectation d'une valeur à une propriété inaccessible
			\item \lstinline!__call ($func, $params)! : est appelée lors de l'appel d'une méthode inaccessible
			\item \lstinline!__isset ()! : est appelée lors du test \lstinline!isset ($o)!
			\item \ldots
		\end{itemize}
	\end{itemize}
\end{frame}

\begin{frame}[containsverbatim]{Les classes : héritage}
	\begin{itemize}
		\item héritage classique, non multiple
		\item implémentation d'interface multiples
		\item la classe mère doit être évidemment définie et chargée avant la classe fille
	\end{itemize}
	\begin{block}{code : héritage}
		\begin{lstlisting}
<?php
abstract class MaClasse { /* ... */ }

class MaClasseFille extends MaClasse { /* ... */ }

interface iIFace1 { /* ... */ }

interface iIFace2 { /* ... */ }

class Classe extends MaClasse implements iIFace1, iIFace2 { /* ... */ } 
		\end{lstlisting}
	\end{block}
\end{frame}


\begin{frame}[containsverbatim]{Les classes : trait (1/2)}
	\begin{itemize}
		\item est un objet regroupant de \textit{manière intéressante} des fonctionnalités
		\item ne s'utilise qu'en inclusion dans des classes ou dans d'autres traits
		\item les méthodes définies dans les \textbf{traits} importés redéfinissent les méthodes héritées, et sont elles-mêmes redéfinies par les méthodes déclarées dans la classe
	\end{itemize}
	\begin{block}{code : traits}
		\begin{lstlisting}
<?php
class Base {
    public function sayHello() {
        echo 'Hello ';
    }
}

trait SayWorld {
    public function sayHello() {
        parent::sayHello();
        echo 'World!';
    }
}

class MyHelloWorld extends Base {
    use SayWorld;
}

$o = new MyHelloWorld();
$o->sayHello();
?>
		\end{lstlisting}
	\end{block}
\end{frame}

\begin{frame}[containsverbatim]{Les classes : trait (2/2)}
	\begin{itemize}
		\item on peut importer plusieurs \textbf{traits} dans une même classe avec \lstinline!use sayHello, sayWorld;!
		\item les conflits de nommages \textbf{doivent} être résolus explicitement~; en choisissant une alternative~: \lstinline!use SayHelloEN, sayHelloFR {sayHelloFR::hello insteadof sayHelloEN ;}!~; en \textit{aliassant} une alternative~: \lstinline!use SayHelloEN, sayHelloFR {sayHelloFR::hello as helloFR ;}!
		\item la visibilité peut également être redéfinie lors de l'utilisation~: \lstinline!use HelloWorld { sayHello as protected; }! ou lors de l'\textit{aliassage}~: \lstinline!use HelloWorld { sayHello as private myPrivateHello; }!
		\item des méthodes statiques et abstraites peuvent être définies dans un \textit{trait}
		\item des variables statiques peuvent être déclarées dans les méthodes définies dans un \textit{trait}
		\item des propriétés peuvent être définies dans un \textit{trait} ; mais cette fois, les collisions de nommage ne peuvent être résolues.
		\item attention, les propriétés statiques définies dans un \textit{trait} n'acquièrent leur caractère partagé qu'une fois importé dans la classe~: si \lstinline!trait staticVar { public static $var ;}! est importé dans les classes \lstinline!classeA! et \lstinline!classeB!, alors \lstinline!classeA::$var <> classeB::$var! est vraie
	\end{itemize}
	\url{https://secure.php.net/manual/fr/language.oop5.traits.php}
\end{frame}

\subsection{Fichiers}

\begin{frame}[containsverbatim]{Scripts}
	\begin{itemize}
		\item \texttt{script} : ensemble du code entre l'appel et la fin d'exécution
		\item extension habituelles : \texttt{.php} (script chapeau), \texttt{.php.inc} (fichiers inclus), \texttt{.ini} (configuration), \texttt{.php.class} (définition de classe), \ldots
		\item le code PHP dans un fichier doit être précédé de \texttt{<?php}
		\item le code PHP dans un fichier doit être suivi de \texttt{?>} ou pas !
		\item on fait référence à un fichier PHP via son chemin dans le système de fichier (et pas l'URL) relativement au chemin du script chapeau ou de façon absolue.
	\end{itemize}
	\begin{block}{fichier.php}
		\begin{lstlisting}
<?php
	echo 'Salut le monde !' ;
		\end{lstlisting}
	\end{block}
\end{frame}

\begin{frame}[containsverbatim]{Inclusions}
	\begin{itemize}
		\item fragmente les gros blocs de code monolithique
		\item regroupe les fonctionnalités
		\item 4 types d'inclusions :
		\begin{itemize}
			\item inclusion simple : \lstinline$include ("fichier.php.inc") ;$ (continue le script si le fichier est manquant, peut être inclus plusieurs fois)
			\item inclusion unique : \lstinline$include_once ("fichier.php.inc") ;$ (continue le script si le fichier est manquant, n'est pas inclus si le fichier l'a déjà été durant le script)
			\item inclusion obligatoire simple : \lstinline$require ("fichier.php.inc") ;$ (le script s'arrête si le fichier est manquant, peut être inclus plusieurs fois)
			\item inclusion obligatoire unique : \lstinline$require_once ("fichier.php.inc") ;$ (le script s'arrête si le fichier est manquant, n'est pas inclus si le fichier l'a déjà été durant le script)
		\end{itemize}
	\end{itemize}
	\begin{block}{inclusions}
		\begin{lstlisting}
<?php
	include ("repertoire/fichier.php.inc") ;
	include_once ("autre_fichier.php.inc") ;
	require ("../autrerepertoire/fichier3.php.inc") ;
	require_once ("/home/user1/fichier4.php.inc") ;
		\end{lstlisting}
	\end{block}
\end{frame}


\subsection{Espaces de nom}

\begin{frame}[containsverbatim]{Espaces de nom, définition}
	Depuis PHP $\ge$ 5.3.0
	\begin{itemize}
		\item encapsulation/nommage d'objets (classes, interface, fonctions, constantes)
		\item éviter les collisions
		\item créer des alias
	\end{itemize}
	\begin{block}{fichier1.php (définition)}
		\begin{lstlisting}
<?php
	namespace MonProjet ;
	
	const CONNEXION_OK = 1 ;
	class Connexion { /* ... */ }
	function connecte() { /* ... */  }
?>
		\end{lstlisting}
	\end{block}
	\begin{block}{fichier2.php (définition sous-espace)}
		\begin{lstlisting}
<?php
	namespace MonProjet\SousEspace ;

	function disconnect () { /* ... */ }
?>
		\end{lstlisting}
	\end{block}
\end{frame}

\begin{frame}[containsverbatim]{Espaces de nom, utilisation}
	\begin{itemize}
		\item nom non qualifié : \texttt{Foo()}
		\item nom qualifié : \texttt{SousEspace\textbackslash{}Foo()}
		\item nom absolu : \texttt{\textbackslash{}MonProjet\textbackslash{}SousEspace\textbackslash{}Foo()}
	\end{itemize}
	\begin{block}{importation et utilisation}
		\begin{lstlisting}
<?php
	use MonProjet ;
	
	echo CONNEXION_OK ;
	$a = new Connexion() ;
	$b = connecte() ;
	SousEspace\disconnect() ;
?>
		\end{lstlisting}
	\end{block}
	\url{https://secure.php.net/manual/fr/language.namespaces.php}
\end{frame}


\subsection{Application}

\begin{frame}{Application PHP}
	\begin{itemize}
		\item pas de notion d'application
		\begin{itemize}
			\item pas d'objet syntaxique particulier
			\item pas de limitation de fichier (\texttt{chroot})
			\item deux applications PHP peuvent interférer
		\end{itemize}
		\item une application PHP est et devrait être définie par le développeur
		\begin{itemize}
			\item configuration (une par application, par exemple)
			\item variables globales
			\item éventuellement un espace de nom global
			\item circonscrite dans un espace de fichier (sécurité : \texttt{chroot}, \texttt{apache}, \ldots)
			\item utilisation d'un script \emph{chapeau}
		\end{itemize}
	\end{itemize}
\end{frame}



	% Exercice Les classes

\section{Les classes}

\begin{frame}[containsverbatim]{Les classes}



\end{frame}


	% Écriture du langage
	% Cours Les règles d’écriture

\section{Règles d’écriture}

\begin{frame}{Pourquoi ?}
\begin{itemize}
\item Fait partie des bonnes pratiques de développement (quelque soit le langage)
\item 
\end{itemize}
\end{frame}

\begin{frame}{Éléments concernés}

\end{frame}

\subsection{Quelques exemples/suggestions de règles}

\begin{frame}{Concernant les espacements}

\end{frame}

\begin{frame}{Concernant l’indentation et les accolades}

\end{frame}

\begin{frame}{Concernant la nomenclature des éléments}

\end{frame}

\begin{frame}{Commentaires et documentation}

\end{frame}

\begin{frame}{Types de données, bibliothèques, versions}

\end{frame}


	%% Exercice Règles d’écriture

	% Utilisation du langage — Les chaînes de caractères
	% Les fonctions str*
	% Utilisation du langage :
% Les chaînes de caractères
%
% Les fonctions str*

\section{Manipulation des chaînes de caractères}

\begin{frame}{Généralités}
	\begin{itemize}
		\item un élément de type \texttt{string} est une suite de caractères traités comme des octets : pas unicode, sauf si l’extension \texttt{mbstrings} est configurée. Dans ce cas, une partie des fonctions de traitement de texte gèrent les caractères multi-octets (UTF-8, notamment) ; d’autres doivent être explicitement informés du codage caractère, les autres ne gèrent pas autre chose que le mono-octet.
		\item \texttt{string} permet la manipulation de chaînes jusqu’à 2147483647 octets (2GiB).
		\item 4 représentations dans le code : 
		\begin{itemize}
			\item \emph{single quote} et \emph{nowdoc} : chaînes sans substitution de variable ni échappement (sauf \textbackslash{}\textbackslash{} et \textbackslash{}\'{})
			\item \emph{double quote} et \emph{heredoc} : chaînes avec substitution de variable et échappement
		\end{itemize}
	\end{itemize}
\end{frame}

\begin{frame}[containsverbatim]{Fonctions de chaînes}
	Constructions :
	\begin{itemize}
		\item Directement avec les 4 représentations.
		\item Par expression :
		\begin{itemize}
			\item Concaténation (\texttt{.} et \texttt{.=}) : \lstinline~$chaine = "chaine " . "autre chaine" ;~ ou \lstinline~$var = 5 ; $chaine = "chaine " . $var ;~
		\end{itemize}
		\item Conversion : \lstinline~strval()~, \lstinline~sprintf()~, \lstinline~print_r()~, etc\ldots
		\item Génération : \lstinline~implode()~, \lstinline~number_format()~, \lstinline~bin2hex()~, \lstinline~str_repeat()~, etc\ldots
	\end{itemize}
	Transformations :
	\begin{itemize}
		\item \lstinline~explode()~, \lstinline~str_replace()~, \lstinline~strtolower()~, \lstinline~strtoupper()~, \lstinline~substr()~, etc\ldots
		\item \lstinline~preg_replace()~, \lstinline~preg_replace_callback()~
	\end{itemize}
		Tests et calculs :
	\begin{itemize}
		\item \lstinline~strlen()~, \lstinline~strpos()~, \lstinline~ord()~, etc\ldots
		\item \lstinline~preg_grep()~, \lstinline~preg_match()~, etc\ldots
	\end{itemize}
	\url{http://fr2.php.net/manual/en/ref.strings.php}\\
	\url{http://fr2.php.net/manual/en/ref.pcre.php}\\
\end{frame}

	
	% Utilisation du langage — Les tableaux
	% Les fonctions array*
	% Utilisation du langage — Les tableaux
% Les fonctions array*

\section{Manipulation des tableaux}

\begin{frame}{Généralités}
	\begin{itemize}
		\item Un objet du type \texttt{array} permet de stocker des paires clé-valeurs de façon ordonnée.
		\item Les clés sont des valeurs scalaires, les valeurs sont de tous types (même des tableaux)
		\item Ainsi les tableaux avec les fonctions adaptées permettent de gérer des structures de données telles que : tableau, liste, file, pile, table de hachage, dictionnaire, arbre, ensemble, tas, etc\ldots
		\item La limite en taille de la structure est celle imposée par l’environnement (système et configuration PHP)
		\item Une des structures les plus utilisées de PHP.
	\end{itemize}
\end{frame}

\begin{frame}[containsverbatim]{Fonctions de chaînes}
	Constructions :
	\begin{block}{construction de tabeau}
		\begin{lstlisting}
<?php
$tableau = array ("un", "deux", 3) ;
$tableau2 = array ("one" => "un", "two" => "deux", "three" => 3) ;
$tableau3 = array (array ("toto" => "titi"), "tutu") ;
$tableau4["tata"] = $tableau3[0] ;
		\end{lstlisting}
	\end{block}
	\begin{itemize}
		\item Expression (\texttt{+}) : \lstinline~$tableau = $t1 + array ("valeur") ;~
		\item Conversion : \lstinline~explode()~, \lstinline~(array)~
		\item Génération : \lstinline~array_diff()~, \lstinline~array_fill()~, \lstinline~array_merge()~, \lstinline~array_slice()~, etc\ldots
	\end{itemize}
	Navigation:
	\begin{itemize}
		\item \lstinline~reset()~, \lstinline~prev()~, \lstinline~next()~, \lstinline~current()~, \lstinline~end()~ 
		\item \lstinline~$tableau["tata"]~, \lstinline~$tableau["tata"][3]~
	\end{itemize}
	Tri, filtre, modification :
	\begin{itemize}
		\item \lstinline~array_filter()~, \lstinline~sort()~, \lstinline~ksort()~, \lstinline~array_walk()~, \lstinline~array_splice()~, etc\ldots
		\item \lstinline~array_pop()~, \lstinline~array_push()~, \lstinline~array_shift()~, \lstinline~array_unshift()~, \lstinline~array_unique()~, etc\ldots
	\end{itemize}
		Tests et calculs :
	\begin{itemize}
		\item \lstinline~count()~, \lstinline~in_array()~, \lstinline~array_key_exists()~, etc\ldots
		\item \lstinline~array_reduce()~, \lstinline~array_search()~, etc\ldots
	\end{itemize}
	\url{http://fr2.php.net/manual/en/ref.array.php}\\
\end{frame}
 

	% Exercices sur les chaînes de caractères et les tableaux.

\section{Exercices : les chaînes et les tableaux}

\begin{frame}[containsverbatim]{Exercice \textnumero{1} : manipulation de chaînes}
	Dans un répertoire dédié (à cette série d’exercices), rapatrier le fichier \texttt{lorem\_ipsum.txt}.
	Pour charger le contenu de ce fichier dans une variable, utiliser le code suivant :
	\begin{block}{chargement d’une chaîne de caractères à partir d’un fichier}
		\begin{lstlisting}
<?php
define ("LOREM_FILE", "./lorem_ipsum.txt") ;

// Si le fichier existe on charge son contenu dans la variable ; sinon, fin du programme.
if (file_exists (LOREM_FILE)) {
	$texte_orig = file_get_contents (LOREM_FILE) ;
} else {
	die ("Le fichier " . LOREM_FILE . " n'est pas disponible.") ;
}
		\end{lstlisting}
	\end{block}
	Programmer :
	\begin{enumerate}
		\item Afficher le contenu de la variable \texttt{\$texte\_orig}
		\item Changer le codage caractère de \texttt{ISO8859-15} à \texttt{UTF-8} et afficher le résultat
		\item Remplacer dans le texte les caractères accentués par leur homologue sans accent et afficher le résultat.
		\item Justifier le texte à 100 caractères (chaque ligne ne doit pas contenir plus de 100 caractères) et afficher le résultat.
		\Pitem Justifier sans couper les mots.
	\end{enumerate}
\end{frame}

\begin{frame}[containsverbatim]{Exercice \textnumero{2} : manipulation de tableaux}
	En reprenant le code de l’exercice précédent\ldots
	Programmer :
	\begin{enumerate}
		\item Insérer chaque ligne du texte justifié comme élément dans un tableau, et afficher le résultat.
		\item Créer un tableau de statistiques répertoriant pour chaque ligne : nombre de mots, nombre de lettre, et nombre d’occurrence de chaque lettre.
		\item Regrouper les statistiques des occurrences des lettres dans un nouveau tableau.
		\item Trier le tableau des statistiques globales, par lettre, l’afficher ; puis par nombre d’occurrences et l’afficher.
	\end{enumerate}
\end{frame}

	
	% Utilisation du langage — Les dates
	% DateTime
	% Utilisation du langage — Les dates

\section{Manipulation des dates}

\begin{frame}{Généralités}
	\begin{itemize}
		\item Les dates et les heures sont gérées nativement par PHP (pas d’extension ou bibliothèque supplémentaire nécessaire)
		\item Les dates générées sont \emph{dépendantes} du serveur sur lequel s’exécute le script (localisation, fuseaux horaires) (\url{http://fr2.php.net/manual/en/datetime.configuration.php})
		\begin{alertblock}{Attention}
			Les dates et heures sont gérées un peu différemment entre $PHP=4$, $PHP<5.2$ et $PHP\ge5.2$ : les bibliothèques entre ces versions ne sont pas totalement compatibles
		\end{alertblock}
		\item Le format interne est codé sur 64 bits, donc permettent de se balader dans le temps de $\pm292$ milliards d’années, environ
		\item Depuis $PHP\ge5.2$, l’objet \texttt{DateTime} est utilisé pour représenter une date absolue ; \texttt{DateTimeZone} est utilisé pour représenter un fuseau horaire (\textit{timezone})
		\item Depuis $PHP\ge5.3$, \texttt{DateInterval} est utilisé pour représenter un intervalle de temps/date ou une date relative ; \texttt{DatePeriod} est utilisé pour itérer sur des dates (dérive de l’interface \texttt{Traversable})
		\item Les dates peuvent également être représentées par \texttt{timestamp} qui est un entier (\texttt{int}) représentant le nombre de secondes depuis \textit{Unix Epoch}, $1^{er}$ janvier 1970 à 00h 00m 00s GMT.
	\end{itemize}
\end{frame}

\begin{frame}{Fonctions}
	Constructions :
	\begin{itemize}
		\item Date absolue : \lstinline~$date = new DateTime ("now", new DateTimeZone ("Europe/Paris")) ;~, \lstinline~$date = DateTime::createFromFormat ("d/m/Y", "12/11/2014") ;~, etc\ldots
		\item Durée : \lstinline~$date_interval = DateInterval::createFromDateString ("1 day") ;~
		\item Liste de dates : \lstinline~foreach (new DatePeriod (new DateTime (), $date_interval, 10) as $d) { /* ... */ }~
		\item Génération d’un \texttt{timestamp} : \lstinline~$timestamp = time () ;~, \lstinline~$timestamp2 = microtime () ;~
	\end{itemize}
	Modifications :
	\begin{itemize}
		\item \lstinline~$date->sub (new DateInterval ("10 days")) ;~, \lstinline~$date->add (new DateInterval ("50 min")) ;~
		\item \lstinline~$date->setTime (23, 59, 59) ;~
	\end{itemize}
	Tests et calculs :
	\begin{itemize}
		\item \lstinline~echo $date->format ("Y-m-d H:i:s") ;~
		\item \lstinline~echo date_sunset($date->getTimestamp (), SUNFUNCS_RET_STRING) ;~
		\item \lstinline~if (checkdate(2, 29, 2001)) \{ /* ... */ \} else \{ /* ... */ \}~
	\end{itemize}
	\url{http://fr2.php.net/manual/en/book.datetime.php}
\end{frame}

	
	% Utilisation du langage — Les fichiers
	% fopen() fclose(), etc…
% 	% Utilisation du langage — Les fichiers

\section{Manipulation des fichiers}

\begin{frame}{Généralités}
	\begin{itemize} 
		\item La manipulation des fichiers sur le système respecte les droits et permissions : l’utilisateur créant le processus (httpd en mode WEB, l’utilisateur lançant la commande php en mode CLI) doit avoir les droits suffisants sur les fichiers à manipuler.
		\item Plusieurs types de fichiers : texte, binaire, ini, json, etc\ldots
		\item Le codage du texte des fichiers doit être maîtrisé : connaître le codage en entrée, et en sortie.
		\item Deux manières de manipuler les fichiers : créer un \textit{handle} (type \texttt{resource}) et manipuler les données à partir cette ressource ; en direct avec les fonctions qui font tout (ouverture, lecture/écriture, fermeture) en une seule opération.
	\end{itemize}
\end{frame}


\begin{frame}{Les fonctions}
	Constructions :
	\begin{itemize}
		\item \lstinline~if (false === file_put_contents ($content, "fichier.txt")) \{ /* erreur */ \}~
		\item \lstinline~$contents = file_get_contents ("fichier.txt") ;~
		\item \lstinline~$file_handle = fopen ("fichier.txt", "a+") ; if (false === fwrite ($file_handle, $content)) \{ /* erreur */ \} fclose ($file_handle) ;~
		\item \lstinline~$file_handle = fopen ("fichier.txt", "r") ; $content = fread ($file_handle, 1024) ; fclose ($file_handle) ;~
	\end{itemize}
	Modifications :
	\begin{itemize}
		\item \lstinline~chmod ("/home/mikael/fichier.txt", 0755) ;~
		\item \lstinline~rename ("/tmp/tmp_file.txt", "/home/user/login/docs/my_file.txt") ;~
	\end{itemize}
	Tests et calculs :
	\begin{itemize}
		\item \lstinline~$rep = dirname ("/etc/passwd") ;~
		\item \lstinline~if (is_writable($filename)) \{ /* ... */ \}~
	\end{itemize}
	\url{http://fr2.php.net/manual/en/ref.filesystem.php}\\
\end{frame}
 
 
	% Exercices sur les dates et les fichiers.

\section{Exercices : les dates et les fichiers}

\begin{frame}[containsverbatim]{Exercice \textnumero{1} : manipulation de dates}
	Programmer :
	\begin{enumerate}
		\item Créer une date représentant votre date de naissance (avec l’heure si vous vous en rappelez \smiley), l’afficher sous différents formats, afficher la date qu’il était à ce moment précis dans d’autre pays (d’autres fuseaux horaires)
		\item Calculer votre âge en secondes, puis minutes, heures, jours, semaines, mois, années, les afficher
		\item Remplir un tableau avec toutes les dates de vos anniversaires, et l’afficher.
	\end{enumerate}
\end{frame}

% \begin{frame}[containsverbatim]{Exercice \textnumero{2} : manipulation de fichiers}
% 	Programmer :
% 	\begin{enumerate}
% 			\item 
% 	\end{enumerate}
% \end{frame}
 

	
	% Utilisation du langage — Les classes
	% Création/Instantiation
	% Méthodes magiques
	% Réflection/Dynamicité
	
	% Utilisation du langage — Manipulation des variables
	% boolval(), intval(), etc…
	% is_bool(), is_string(), etc…
	% isset(), unset()
	% serialize(), unserialize()
	% print_r(), var_dump(), var_export()
	
	% Utilisation du langage — Les bases de données
	% Accès aux bases
	
	% Utilisation du langage — Autres fonctions
	% exit(), die(), eval(), etc…
	% constant(), define(), defined()
	% sleep(), time_nanosleep(), time_sleep_until(), usleep()
	
	% Utilisation du langage — D’autres bibliohèques
	% SPL
	% SOAP
	% JSON
	
	% Utilisation du langage — Données HTML
	% Fabriquer des formulaires
	% Interpréter les données de formulaire
	
	% Utilisation du langage — Affichage
	% sorties standards
	% objets output buffering ob_*
	
	% Utilisation du langage — Sessions
	
	% Utilisation du langage — Exceptions
	
	% Utilisation du langage — Débogage
	% XDebug
	% APD
	
	% Architecture d’application (MVC, BO/DAO/PO)
	
	% Outils de développement en entreprise
	
\end{document}

% Développer avec PHP5 en entreprise, sur 10 cours
% Mikaël Cordon
% 2014-10-06

% Document maître.

\documentclass{beamer}
% Replace the \documentclass declaration above
% with the following two lines to typeset your 
% lecture notes as a handout:
%\documentclass{article}
%\usepackage{beamerarticle}
\usepackage[utf8]{inputenc}
\usepackage[french]{babel}
\usepackage{textcomp}

%\usetheme{Montpellier}
\usetheme{Antibes}

\title{PHP~5}

% A subtitle is optional and this may be deleted
\subtitle{en environnement professionnel}

\author{Mikaël~Cordon}

\date{Université de Poitiers, LP TLWTM, 2014-2015}

\subject{Université Poitiers, LP TLWTM, 2014-2015 -- PHP 5}

% \pgfdeclareimage[height=0.5cm]{university-logo}{university-logo-filename}
% \logo{\pgfuseimage{university-logo}}

\AtBeginSection[]
{
  \begin{frame}<beamer>{Vous êtes ici !}
    \tableofcontents[sectionstyle=show/shaded,subsectionstyle=show/hide/hide,subsubsectionstyle=hide]
  \end{frame}
}
\AtBeginSubsection[]
{
  \begin{frame}<beamer>{Vous êtes ici !}
    \tableofcontents[sectionstyle=show/hide,subsectionstyle=show/shaded/hide,subsubsectionstyle=show/hide/hide/hide]
  \end{frame}
}

\begin{document}

\begin{frame}
  \titlepage
\end{frame}

\begin{frame}{Sommaire}
  \tableofcontents[subsectionstyle=hide,subsubsectionstyle=hide]
\end{frame}

\section{Cours \textnumero{}1 : présentation de PHP}
\subsection{Présentation}
\begin{frame}{Historique}
    \begin{itemize}
	    \item 1994 : Rasmus Lerdorf, \og~Personal Home Page / Forms Interpreter~\fg
    	\item \og~PHP : Hypertext Preprocessor~\fg 
    	\item PHP~4 (2000) : version très utilisée (autrefois), \textbf{n’est pas} orientée objet
    	\item PHP~5 (2004) : version de PHP \textbf{orientée objet}
    	\item version actuelle (2014-08) : 5.6
    \end{itemize}
\end{frame}

\begin{frame}{Qu'est-ce que PHP ?}
    \begin{block}{\url{http://php.net/}}PHP is a popular general-purpose scripting language that is especially suited to web development.\\Fast, flexible and pragmatic, PHP powers everything from your blog to the most popular websites in the world.
    \end{block}
    \begin{itemize}
        \item Logiciel libre : PHP license v3.01 (BSD-style) ; Documentation : CC-By PHP Documentation Group v3
        \item langage à script : interprétation à l’exécution
        \item langage de haut niveau : accès fichiers, sessions, caches, protocole HTTP, les codages caractères (UTF-8)\ldots
        \item beaucoup de bibliothèques disponibles : PEAR, PECL (extensions)
    \end{itemize}
    Documentation de référence : \url{http://www.php.net/manual}
\end{frame}

\begin{frame}{Technologies associées}
	Rarement utilisé seul, PHP s’associe, via des bibliothèques et/ou nativement, à :
	\begin{itemize}
		\item httpd : apache, nginx, lighttpd, IIS, \ldots
		\item bases de données : MySQL/MariaDB, PostgreSQL, Oracle, DB2, \ldots
		\item présentation : html (html-4.0, XHTML, XML, HTML5), css, javascript, templates, \ldots
		\item communications : fichiers, flux, http/s, ftp/s, \ldots
		\item \ldots
	\end{itemize}
\end{frame}

\subsection{Architecture de PHP}

\begin{frame}{Fonctionnement WEB : client/serveur}
	\begin{itemize}
		\item[1] client : le navigateur WEB (Firefox, Chromium, Lynx, IE, \ldots)
			\begin{itemize}
				\item effectue des requêtes : URL (requête HTTP + commande GET, POST, PUT, DELETE)
				\item interprète la réponse reçue : HTML, fichier, CSS, javascript, erreurs (404, 500, 501, \ldots) 
			\end{itemize} 
		\item[2] serveur : httpd 
			\begin{itemize}
				\item httpd ouvre, décode éventuellement la session HTTPS
				\item httpd interprète les URL et appelle le script PHP
				\item PHP exécute le script et génère une réponse HTTP : header HTTP + document
			\end{itemize}
		\item protocole de communication : HTTP 1.1 (éventuellement encapsulé dans un tunnel SSL)
		 	\begin{itemize}
		 		\item transporte la requête HTTP (URL + entêtes)
		 		\item transporte la réponse HTTP (entêtes + document)
		 	\end{itemize}
	\end{itemize}
\end{frame}

\begin{frame}{Installation}
	\begin{itemize}
		\item PHP est un exécutable ; les permissions et droits de l'utilisateur exécutant et ceux sur les fichiers et l'exécutable s'appliquent :
			\begin{itemize}
				\item l'utilisateur et l'environnement de httpd en mode WEB
				\item l'utilisateur courant et son environnement en mode CLI
			\end{itemize}
		\item PHP est livré avec des bibliothèques et des fichiers de configuration
	\end{itemize}
	Installation :
	\begin{itemize}
		\item GNU/Linux : paquet \texttt{php} directement depuis le dépôt (\texttt{.deb}, \texttt{.rpm}, \texttt{.tgz}, \ldots) ; penser aux paquets \texttt{libapachemod-php5}, \ldots
		\item Microsoft Windows : \texttt{WAMP} (Windows Apache MySQL PHP)
		\item Apple MACOSX : \texttt{MAMP} (Mac Apache MySQL PHP), \texttt{AMPPS}, \ldots
	\end{itemize}
\end{frame}

\begin{frame}{Configuration}
	Plusieurs notions :
	\begin{itemize}
		\item environnement : prod, dev, test, \ldots
		\item fonctionnalités : dés/activer des extensions, type et niveau d'erreurs, \ldots
		\item paramétrage : accès mail, accès bdd, \ldots
		\item sécurité : exposition de données sensibles, niveaux de chiffrages, hachage, \ldots
	\end{itemize}
	Plusieurs niveaux :
	\begin{itemize}
		\item global : \texttt{php.ini} (\texttt{/etc/php/})
		\item utilisateur : \texttt{.user.ini} (\texttt{\textasciitilde/}, \texttt{\$HOME/})
		\item service : \texttt{php.ini} (\texttt{/etc/apache/conf.d/})
		\item projet : configuration httpd (\texttt{apache.conf}, \texttt{.htaccess})
		\item on-the-fly : \texttt{ini\_get()} et \texttt{ini\_set()}
	\end{itemize}
\end{frame}

\begin{frame}{Éléments de configuration importants}
	\begin{itemize}
		\item max\_memory
		\item maxfilesize
		\item maxuploadsize
	\end{itemize}
\end{frame}

\section{Exercice \textnumero{}1 : \texttt{Hello world!}}

\begin{frame}{Hello world!}
\end{frame}

\section{Cours \textnumero{}2 : Les éléments du langage}

%\subsection{Éléments du langage}

%\begin{frame}{Les éléments}
%\begin{itemize}
%\item application ou programme : abstrait, regroupant le reste.
%\item espaces de nom : organise le code et évite les collisions
%\item fichiers : scripts, inclusions, définitions, configurations
%\item classes : déjà définies ou à définir
%\item fonctions ou clôtures : déjà définies ou à définir
%\item structures du langage (\texttt{if}, \texttt{while}, \texttt{switch}, \texttt{\{\}}) : déjà définies
%\item instructions : le code
%\item commentaires et documentation : pour le développeur
%\item variables : \emph{références} à des valeurs
%\item types de valeur (classes) : typage \emph{faible}
%\item valeurs : les données
%\item constantes : valeurs définies une seule fois
%\end{itemize}
%\end{frame}

\subsection{Application}

\begin{frame}{Application PHP}

\end{frame}

\subsection{Espaces de nom}

\begin{frame}{Espaces de nom}

\end{frame}

\subsection{Fichiers}

\begin{frame}{Scripts}

\end{frame}

\begin{frame}{Inclusions}

\end{frame}

\begin{frame}{Fichiers de définition de classe}

\end{frame}

\begin{frame}{Fichiers \texttt{.ini}}

\end{frame}

\subsection{Classes}

\begin{frame}{Classes \guillemotleft~normales~\guillemotright}

\end{frame}

\begin{frame}{Méthodes}

\end{frame}

\begin{frame}{Données de classe}

\end{frame}

\begin{frame}{Instanciation, libération}

\end{frame}

\begin{frame}{Visibilité}

\end{frame}

\begin{frame}{Redéfinition et surcharge}

\end{frame}

\begin{frame}{Héritage}

\end{frame}

\begin{frame}{Classes abstraites}

\end{frame}

\begin{frame}{Interface}

\end{frame}

\subsection{Fonctions, clôtures}

\begin{frame}{Fonctions} %function name()

\end{frame}

\begin{frame}{Paramètres}

\end{frame}

\begin{frame}{Fonctions anonymes} %function ()

\end{frame}

\begin{frame}{Clôtures de code} %traits

\end{frame}

\subsection{Contrôle de l’exécution}

\begin{frame}{Exécution conditionnelle} %if

\end{frame}

\begin{frame}{Exécution en boucle} %while

\end{frame}

\begin{frame}{Exécution sélective} %switch

\end{frame}

\begin{frame}{Regroupement en bloc anonyme} %{}

\end{frame}

\begin{frame}{Exceptions} % try

\end{frame}

\subsection{Instructions}

\begin{frame}{Instruction}

\end{frame}

\begin{frame}{Affectation}

\end{frame}

\begin{frame}{Appels}

\end{frame}

\subsection{Commentaires et documentation}

\begin{frame}{Commentaires}

\end{frame}

\begin{frame}{Documentation}

\end{frame}

\subsection{Variables}

\begin{frame}{Variable}

\end{frame}

\begin{frame}{Référence}

\end{frame}

\subsection{Types de valeurs, valeurs et constantes}

\begin{frame}{Types de valeurs}
Les scalaires :
\begin{itemize}
\item Entiers relatifs : \texttt{int}
\item Flottants : \texttt{float}
\item Chaîne de caractère : \texttt{string}
\item Booléen : \texttt{bool}
\end{itemize}
Les non scalaires :
\begin{itemize}
\item Tableaux : \texttt{array}
\item Objet : \texttt{object}
\end{itemize}
Le pseudo-type \texttt{null}.
\end{frame}

\begin{frame}{Entier}

\end{frame}

\begin{frame}{Flottant}

\end{frame}

\begin{frame}{Chaîne de caractères}

\end{frame}

\begin{frame}{Booléen}

\end{frame}

\begin{frame}{Tableau}

\end{frame}

\begin{frame}{Objet}

\end{frame}

\begin{frame}{Sans type}

\end{frame}

\section{Règles d’écriture}

\begin{frame}{Pourquoi ?}

\end{frame}

\begin{frame}{Éléments concernés}

\end{frame}

\subsection{Quelques exemples/suggestions de règles}

\begin{frame}{Concernant les espacements}

\end{frame}

\begin{frame}{Concernant l’indentation et les accolades}

\end{frame}

\begin{frame}{Concernant la nomenclature des éléments}

\end{frame}

\begin{frame}{Commentaires et documentation}

\end{frame}

\begin{frame}{Types de données, bibliothèques, versions}

\end{frame}

\section{Technologies annexes}

\subsection{Sessions}

\subsection{Authentification}

\subsection{Caches}

\subsection{Codage caractères}

\end{document}

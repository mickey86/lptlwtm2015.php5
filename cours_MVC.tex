% Architecture d’application (MVC, BO/DAO/PO)

\section{Architecture MVC (DAO, PO, BO)}

\begin{frame}[containsverbatim]{Généralités}
	\begin{itemize}
		\item organisation : en couches, en modules, en parties, etc\ldots
		\item force la modularisation, l'architecture
		\item indépendance technologique, technique, des développements, de la maintenance
		\item pour les acteurs d'un projet MVC : naviguer rapidement, repérer les bogues, appliquer des schémas standards, importer des briques d'autres projets, sûreté de développement, etc\ldots
		\item pour les gros projets : chacun son métier, plusieurs sous-projets, etc\ldots
		\item pour les projets plus modestes : assemblage rapide, environnements de dév adaptés, documentation, bibliothèques (MVC) disponibles et adaptées
		\item MVC : Model, View, Controler ; Modèle, Vue, Contrôleur
		\item BO/PO/DAO(/TO) : Business Objects, Presentation Objects, Data Abstraction Objects (, Tools Objects)
		\item vue $\leftrightarrow$ contrôleur $\leftrightarrow$ modèle
	\end{itemize}
\end{frame}

\begin{frame}[containsverbatim]{Modèle, DAO (Data Abstraction Objects) (1/3)}
	Comprend :	
	\begin{itemize}
		\item le modèle de données : description des entités, des propriétés, des contraintes, des multiplicités, etc\ldots
		\item l'accès aux données : les requêtes, CRUD + spécialisés
		\item CRUD : Create, Read, Update, Delete
	\end{itemize}
	Les entités, typiquement :	
	\begin{itemize}
		\item une classe $\leftrightarrow$ une entité ; une instance de classe $\leftrightarrow$ une instance de l'entité
		\item la classe décrit les propriétés, certaines contraintes, les multiplicités
		\item l'instance stocke une donnée de l'entité : un ensemble des valeurs des propriétés, éventuellement référence vers d'autres entités
	\end{itemize}
	Les accès (persistance), typiquement :	
	\begin{itemize}
		\item une connexion à la base de données + des requêtes
		\item une classe de connexion générique à la base de données (classe PDO ou dérivée)
		\item une classe d'accès par entité ; les méthodes sont les requêtes
	\end{itemize}
\end{frame}

\begin{frame}[containsverbatim]{Modèle, DAO (Data Abstraction Objects) (2/3)}
	\begin{block}{exemple : DAO/Utilisateur.php.class *}
		\begin{lstlisting}
<?php class Utilisateur {
	private $id = null ;
	private $login = null ;
	private $tags = null ;
	public function __construct ($id = null, $login, array $tags = array()) {
		$this->id = $id ;
		$this->login = $login ;
		$this->tags = $tags ;
	}
	public function get_id () { 
		return $this->id ; 
	}
	public function set_id ($id) { 
		$this->id = $id ; 
	}
	public function get_tags () { 
		return $this->tags ; 
	}
	public function set_tags (array $tags) { 
		$this->tags = $tags ; 
	}
	...
}
		\end{lstlisting}
	\end{block}
	* Ici, pour des questions évidentes de place, il n'y a pas de documentation ; en vrai, tout doit être documenté !
\end{frame}

\begin{frame}[containsverbatim]{Modèle, DAO (Data Abstraction Objects) (3/3)}
	\begin{block}{exemple : DAO/Utilisateur\_DAO\_Impl.php.class *}
		\begin{lstlisting}
<?php require_once ("DAO/DAO.php.class") ;
require_once ("DAO/Utilisateur_DAO.php.class") ;
require_once ("DAO/Tag_DAO.php.class") ;
require_once ("DAO/Utilisateur.php.class") ;
class Utilisateur_DAO_Impl extends DAO implements Utilisateur_DAO {
	private $tag_DAO ;
	public function __construct (PDO $pdo, DAO $tag_DAO = null) {
		parent::__construct ($pdo) ;
		$this->tag_DAO = $tag_DAO;
	}
	public function findById($id) {
		try {
			$statement = $this->get_pdo()->prepare("SELECT * FROM Users WHERE id = :id") ;
			$result = $statement->execute(array("id" => $id)) ;
		} catch (PDOException $pdoe) {
			log(__CLASS__ . " : " . print_r($pdoe, true)) ;
			throw $pdoe ;
		}
		if ($row = $result->fetch()) {
			$utilisateur = new Utilisateur($row["id"], $row["login"]) ;
			if ($this->tag_DAO) {
				$utilisateur->set_tags(
					$this->tag_DAO->findAllByUtilisateur($utilisateur)) ;
			}
			return $utilisateur ;
		} else {
			return null ;
		}
	}
}
		\end{lstlisting}
	\end{block}
	* Ici, pour des questions évidentes de place, il n'y a pas de documentation ; en vrai, tout doit être documenté !
\end{frame}

\begin{frame}[containsverbatim]{Vue, PO (Presentation Objects) (1/2)}
	Contient :
	\begin{itemize}
		\item ce qui est exposé aux utilisateurs (humain ou non) de l'application
		\item présentation des données (textes, images, vidéos, audios, tableaux, fichiers, etc\ldots) via HTML ou fichiers, par exemple
		\item formulaires d'entrée de données (textes, fichiers, cases à cocher, listes de sélections, etc\ldots) via HTML ou fichiers, HTTP GET, HTTP POST, HTTP PUT, par exemple
		\item interface d'accès aux fonctionnalités (déclenchement de procédures, d'actions) via HTML, Javascript, HTTP GET, HTTP DELETE, par exemple
		\item les classes de contruction des formulaires et d'envoi de données : met en forme les données (qu'elle récupère auprès de la couche métier, ou qui sont fournies lors de l'appel)
		\item les classes de \og déconstruction \fg{} de formulaires et des données venant du client : extrait les données pour les fournir en bonne et due forme à la couche métier
	\end{itemize}
	Les données et les actions :
	\begin{itemize}
		\item seules les données fournies par la couche métier ne peuvent être présentées
		\item la couche de présentation HTML ne doit pas être considérée comme fiable ; surtout en ce qui concerne les données entrées par l'utilisateur (même en HTTPS !) 
	\end{itemize}
	La présentation peut analyser l'environnement client (via javascript, par exemple) pour adapter la présentation des données (conjointement ou non avec la couche métier) : internationalisation (langue, régions, codage caractères), adaptation au terminal (PC, mobile, imprimante, autre), accessibilité, etc\ldots
\end{frame}

\begin{frame}[containsverbatim]{Vue, PO (Presentation Objects) (2/2)}
	\begin{block}{exemple : PO/Utilisateurs\_Liste\_Page.php.class *}
		\begin{lstlisting}
<?php require_once ("PO/HTML_renderer.php.class") ;
include_once ("PO/Default_Templates/Template.php.class") ;
class Utilisateurs_Liste_Page extends Presentation_Table {
	private $utilisateurs_data = null ;
	private $template = null ;
	public function __construct (Template $template, Renderer $renderer) {
		parent::__construct($renderer) ;
		$this->template = $template ;
	}
	public function get_data ($index) { 
		return $this->data[$index] ; 
	}
	public function set_data ($index, array $data) { 
		$this->data[$index] = $data ; 
	}
	public function display () {
		$data_to_render = array() ;
		foreach ($this->data as $index => $u_data) {
			$data_to_render["table_rows"][] = array (
				"numero" => $index + 1,
				"login" => strtolower($u_data["login"]),
				"tags" => implode (", ", $u_data["tags"]),
			) ;
		}
		$data_to_render["table_title"] = "Liste des utilisateurs" ;
		$data_to_render["table_size"] = $config["default_table_size"] ;
		echo $this->get_renderer()->render($this->template, $data_to_render) ;
	}
	...
}
		\end{lstlisting}
	\end{block}
	* Ici, pour des questions évidentes de place, il n'y a pas de documentation ; en vrai, tout doit être documenté !
\end{frame}

\begin{frame}[containsverbatim]{Contrôleur, BO (Business Objects) (1/3)}
	Contient :
	\begin{itemize}
		\item l'intelligence de l'application
		\item traitement des données
		\item gestion des autorisations, des accès, sécurité, etc\ldots
		\item gestion des workflows (enchaînement d'étapes) : inscriptions, commandes, etc\ldots
		\item gestion des sessions
	\end{itemize}
	En général :
	\begin{itemize}
		\item doit être le plus indépendant possible des technologies des couches de présentation et de stockage de données : pas de HTML, pas de CSS, pas de javascript, pas de SQL,
		\item ne doit exposer aux autres couches que les fonctionnalités qui leur sont dédiées
	\end{itemize}
\end{frame}

\begin{frame}[containsverbatim]{Contrôleur, BO (Business Objects) (2/3)}
	\begin{block}{exemple : index.php *}
		\begin{lstlisting}
<?php define ("CONFIG_FILENAME", "configuration.ini") ;
if (file_exists(CONFIG_FILENAME)) {
	$CONFIG = parse_ini("configuration.ini") ;
	require_once ("BO/main.php.inc") ;	
} else 	die ("Fichier de configuration manquant.") ;
		\end{lstlisting}
	\end{block}
	\begin{block}{exemple : BO/main.php.inc *}
		\begin{lstlisting}
<?php require_once ("PO/ClientHTTP.php.class") ;
session_start () ;
if (isset ($_SERVER)) {
	$client = new ClientHTTP () ; 
	if ($_SERVER["REQUEST_METHOD"] == "GET" && isset ($_GET["q"])) {
		switch ($_GET["q"]) {
		case "listusers" :
			require_once ("BO/Lister_Utilisateurs_Cmd.php.class") ;
			require_once ("PO/Utilisateurs_Liste_Page.php.class") ;
			$commande = new Lister_Utilisateur_Cmd () ;
			$commande->execute () ;
			$page = new Utilisateur_Liste_Page () ;
			$page->set_data ($commande->get_outdata ()) ;
			$client->send ($page->display ()) ;
		...
		}
	} else {
		$client->send ("Erreur de commande.", 400) ;
	}
}
session_write_close () ;
		\end{lstlisting}
	\end{block}
	* Ici, pour des questions évidentes de place, il n'y a pas de documentation ; en vrai, tout doit être documenté !
\end{frame}

\begin{frame}[containsverbatim]{Contrôleur, BO (Business Objects) (3/3)}
	\begin{block}{exemple : BO/Lister\_Utilisateurs\_Cmd.php.class *}
		\begin{lstlisting}
<?php require_once ("BO/Command.php.class") ;
require_once ("DAO/Utilisateur.php.class") ;
require_once ("DAO/Utilisateur_DAO_Impl.php.class") ;
class Lister_Utilisateurs_Cmd implements Command {
	private $outdata = array () ;
	private $Utilisateur_DAO = null;
	...
	public function execute () {
		$utilisateurs = $this->Utilisateur_DAO->findAll() ;
		$i = 0 ;
		foreach ($utilisateurs as $u) {
			$outdata[] = array (
				"numero" => $i ++, 
				"login" => $u->get_login (), 
				"tags" => $u->get_tags (),
			) ; 
		}
	}
}
		\end{lstlisting}
	\end{block}
	* Ici, pour des questions évidentes de place, il n'y a pas de documentation ; en vrai, tout doit être documenté !
\end{frame}


% Utilisation du langage — Les Sessions

\section{Les sessions}

\begin{frame}[containsverbatim]{Généralités}
	\begin{itemize}
		\item Permet de conserver des données d’un appel de script sur l’autre sous condition que PHP aie pu appairer la session demandée par le navigateur avec une session déjà enregistrée.
		\item Plusieurs mécanismes permettent d’identifier une session :
		\begin{itemize}
			\item Passage d’un numéro de session au navigateur dans les données HTTP, et récupération du numéro de session dans les données \texttt{GET} (URL) : très mauvaise idée !
			\item Utilisation d’un cookie de session : plus sécurisé, pris en charge par PHP (il faut l’activer dans la configuration : \texttt{session.use\_cookies=1} et \texttt{session.use\_only\_cookies=1})
		\end{itemize}
		\item Pour re/démarrer une session et éventuellement la réinitialiser :
		\begin{block}{session}
			\begin{lstlisting}
<?php
// Surtout aucun envoi de donnee ou d'entete au navigateur avant session_start
session_start () ;
if (isset ($_SESSION["compte"])) {
	$_SESSION["compte"] ++ ;
} else {
	$_SESSION["compte"] = 1 ;
}
if (isset ($_SESSION["reset"]) && ! empty ($_SESSION["reset"])) {
	session_destroy () ;
	session_restart () ;
}
// Script
session_write_close () ;
			\end{lstlisting}
		\end{block}
	\end{itemize}
\end{frame}

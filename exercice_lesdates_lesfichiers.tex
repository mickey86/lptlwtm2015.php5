% Exercices sur les dates et les fichiers.

\section{Exercices : les dates et les fichiers}

\begin{frame}[containsverbatim]{Exercice \textnumero{1} : manipulation de dates}
	Programmer :
	\begin{enumerate}
		\item Créer une date représentant votre date de naissance (avec l’heure si vous vous en rappelez \smiley), l’afficher sous différents formats, afficher la date qu’il était à ce moment précis dans d’autre pays (d’autres fuseaux horaires)
		\item Calculer votre âge en secondes, puis minutes, heures, jours, semaines, mois, années, les afficher
		\item Remplir un tableau avec toutes les dates de vos anniversaires, et l’afficher.
	\end{enumerate}
\end{frame}

\begin{frame}[containsverbatim]{Exercice \textnumero{2} : manipulation de fichiers}
	Programmer :
	\begin{enumerate}
		\item Reprendre le tableau des dates d’anniversaire, les ranger dans un fichier CSV (Comma Separated Values) avec pour champs : le rang, la date en France dans un format, puis le nom d’un autre pays, et la date dans cet autre pays.
			\item Recharger les données dans un tableau associatif (avec des noms de clé qui vont bien) : les dates en \texttt{DateTime}, rang en \texttt{int} et texte en \texttt{string}
	\end{enumerate}
\end{frame}
 

% Cours № 2

\section{Cours \textnumero{}2 : Les éléments du langage}

%\subsection{Éléments du langage}

%\begin{frame}{Les éléments}
%\begin{itemize}
%\item application ou programme : abstrait, regroupant le reste.
%\item espaces de nom : organise le code et évite les collisions
%\item fichiers : scripts, inclusions, définitions, configurations
%\item classes : déjà définies ou à définir
%\item fonctions ou clôtures : déjà définies ou à définir
%\item structures du langage (\texttt{if}, \texttt{while}, \texttt{switch}, \texttt{\{\}}) : déjà définies
%\item instructions : le code
%\item commentaires et documentation : pour le développeur
%\item variables : \emph{références} à des valeurs
%\item types de valeur (classes) : typage \emph{faible}
%\item valeurs : les données
%\item constantes : valeurs définies une seule fois
%\end{itemize}
%\end{frame}

\subsection{Application}

\begin{frame}{Application PHP}

\end{frame}

\subsection{Espaces de nom}

\begin{frame}{Espaces de nom}

\end{frame}

\subsection{Fichiers}

\begin{frame}{Scripts}

\end{frame}

\begin{frame}{Inclusions}

\end{frame}

\begin{frame}{Fichiers de définition de classe}

\end{frame}

\begin{frame}{Fichiers \texttt{.ini}}

\end{frame}

\subsection{Classes}

\begin{frame}{Classes \guillemotleft~normales~\guillemotright}

\end{frame}

\begin{frame}{Méthodes}

\end{frame}

\begin{frame}{Données de classe}

\end{frame}

\begin{frame}{Instanciation, libération}

\end{frame}

\begin{frame}{Visibilité}

\end{frame}

\begin{frame}{Redéfinition et surcharge}

\end{frame}

\begin{frame}{Héritage}

\end{frame}

\begin{frame}{Classes abstraites}

\end{frame}

\begin{frame}{Interface}

\end{frame}

\subsection{Fonctions, clôtures}

\begin{frame}{Fonctions} %function name()

\end{frame}

\begin{frame}{Paramètres}

\end{frame}

\begin{frame}{Fonctions anonymes} %function ()

\end{frame}

\begin{frame}{Clôtures de code} %traits

\end{frame}

\subsection{Contrôle de l’exécution}

\begin{frame}{Exécution conditionnelle} %if

\end{frame}

\begin{frame}{Exécution en boucle} %while

\end{frame}

\begin{frame}{Exécution sélective} %switch

\end{frame}

\begin{frame}{Regroupement en bloc anonyme} %{}

\end{frame}

\begin{frame}{Exceptions} % try

\end{frame}

\subsection{Instructions}

\begin{frame}{Instruction}

\end{frame}

\begin{frame}{Affectation}

\end{frame}

\begin{frame}{Appels}

\end{frame}

\subsection{Commentaires et documentation}

\begin{frame}{Commentaires}

\end{frame}

\begin{frame}{Documentation}

\end{frame}

\subsection{Variables}

\begin{frame}{Variable}

\end{frame}

\begin{frame}{Référence}

\end{frame}

\subsection{Types de valeurs, valeurs et constantes}

\begin{frame}{Types de valeurs}
Les scalaires :
\begin{itemize}
\item Entiers relatifs : \texttt{int}
\item Flottants : \texttt{float}
\item Chaîne de caractère : \texttt{string}
\item Booléen : \texttt{bool}
\end{itemize}
Les non scalaires :
\begin{itemize}
\item Tableaux : \texttt{array}
\item Objet : \texttt{object}
\end{itemize}
Le pseudo-type \texttt{null}.
\end{frame}

\begin{frame}{Entier}

\end{frame}

\begin{frame}{Flottant}

\end{frame}

\begin{frame}{Chaîne de caractères}

\end{frame}

\begin{frame}{Booléen}

\end{frame}

\begin{frame}{Tableau}

\end{frame}

\begin{frame}{Objet}

\end{frame}

\begin{frame}{Sans type}

\end{frame}

\section{Règles d’écriture}

\begin{frame}{Pourquoi ?}

\end{frame}

\begin{frame}{Éléments concernés}

\end{frame}

\subsection{Quelques exemples/suggestions de règles}

\begin{frame}{Concernant les espacements}

\end{frame}

\begin{frame}{Concernant l’indentation et les accolades}

\end{frame}

\begin{frame}{Concernant la nomenclature des éléments}

\end{frame}

\begin{frame}{Commentaires et documentation}

\end{frame}

\begin{frame}{Types de données, bibliothèques, versions}

\end{frame}


% Utilisation du langage — Les dates

\section{Manipulation des dates}

\begin{frame}{Généralités}
	\begin{itemize}
		\item Les dates et les heures sont gérées nativement par PHP (pas d’extension ou bibliothèque supplémentaire nécessaire)
		\item Les dates générées sont \emph{dépendantes} du serveur sur lequel s’exécute le script (localisation, fuseaux horaires) (\url{http://fr2.php.net/manual/en/datetime.configuration.php})
		\begin{alertblock}{Attention}
			Les dates et heures sont gérées un peu différemment entre $PHP=4$, $PHP<5.2$ et $PHP\ge5.2$ : les bibliothèques entre ces versions ne sont pas totalement compatibles
		\end{alertblock}
		\item Le format interne est codé sur 64 bits, donc permettent de se balader dans le temps de $\pm292$ milliards d’années, environ
		\item Depuis $PHP\gt5.2$, l’objet \texttt{DateTime} est utilisé pour représenter une date absolue ; \texttt{DateTimeZone} est utilisé pour représenter un fuseau horaire (\textit{timezone})
		\item Depuis $PHP\ge5.3$, \texttt{DateInterval} est utilisé pour représenter un intervalle de temps/date ou une date relative ; \texttt{DatePeriod} est utilisé pour itérer sur des dates (dérive de l’interface \texttt{Traversable})
		\item Les dates peuvent également être représentées par \texttt{timestamp} qui est un entier (\texttt{int}) représentant le nombre de secondes depuis \textit{Unix Epoch}, $1^{er}$ janvier 1970 à 00h 00m 00s GMT.
	\end{itemize}
\end{frame}

\begin{frame}[containsverbatim]{Fonctions}
	Constructions :
	\begin{itemize}
		\item Date absolue : \lstinline§$date = new DateTime ("now", new DateTimeZone ("Europe/Paris")) ;§, \lstinline§$date = DateTime::createFromFormat ("d/m/Y", "12/11/2014") ;§, etc\ldots
		\item Durée : \lstinline§$date_interval = DateInterval::createFromDateString ('1 day') ;§
		\item Liste de dates : \lstinline§foreach (new DatePeriod (new DateTime (), $date_interval, 10) as $d) { /* ... */ }§
		\item Génération d’un timestamp : \lstinline§$timestamp = time () ;§, \lstinline§$timestamp2 = microtime () ;§
	\end{itemize}
	Modifications :
	\begin{itemize}
		\item \lstinline§$date->sub (new DateInterval ("10 days")) ;§, \lstinline§$date->add (new DateInterval ("50 min")) ;§, 
		\item \lstinline§$date->setTime (23, 59, 59) ;§
	\end{itemize}
	Tests et calculs :
	\begin{itemize}
		\item \lstinline§echo $date->format ("Y-m-d H:i:s") ;§
		\item \lstinline§echo date_sunset($date->getTimestamp (), SUNFUNCS_RET_STRING) ;§
		\item \lstinline§if (checkdate(2, 29, 2001)) { /* ... */ } else { /* ... */ }§
	\end{itemize}
	\url{http://fr2.php.net/manual/en/book.datetime.php}
\end{frame}

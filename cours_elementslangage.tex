% Cours Les éléments du langage

\section{Les éléments du langage}

\subsection{Types de valeurs, valeurs et constantes}

\begin{frame}{Types de valeurs}
	Les scalaires :
	\begin{itemize}
		\item Entiers relatifs : \texttt{int}
		\item Flottants : \texttt{float}
		\item Chaîne de caractères : \texttt{string}
		\item Booléen : \texttt{bool}
	\end{itemize}
	Les non scalaires :
	\begin{itemize}
		\item Tableau : \texttt{array}
		\item Objet : \texttt{object}
		\item Ressource : \texttt{resource}
	\end{itemize}
	Le pseudo-type \texttt{null}
\end{frame}

\begin{frame}[containsverbatim]{Entier}
	\begin{itemize}
		\item entier relatif : $[-2^{31}$ ; $2^{31} - 1]$ ou  $[-2^{63}$ ; $2^{63} - 1]$ (32 ou 64 bits selon l'architecture système)
		\item est pris automatiquement pris pour un entier si seul dans une chaîne de caractères
		\item peut être représenté dans différentes bases (2, 8, 10, 16) 
	\end{itemize}
	\begin{block}{code : entiers}
		\begin{lstlisting}
<?php
	$a = 1234 ; // un nombre decimal
	$a = -123 ; // un nombre negatif
	$a = 0123 ; // un nombre octal (equivalent a 83 en decimal)
	$a = 0x1A ; // un nombre hexadecimal (equivalent a 26 en decimal)
	$a = 0b11111111 ; // un nombre binaire (equivalent a 255 en decimal)
		\end{lstlisting}
	\end{block}
\end{frame}

\begin{frame}[containsverbatim]{Flottant}
	\begin{itemize}
		\item nombres à partie décimale (14 chiffres de précision) sur 64 ou 128 bits selon l'architecture système
		\item environ $1,8 \times 10^{308}$ est un maximum fréquent
		\item s'écrit avec la convention anglosaxone (point et pas virgule pour le séparateur décimal)  
	\end{itemize}
	\begin{block}{code : flottants}
		\begin{lstlisting}
<?php
	$a = 1.234 ;
	$b = 1.2e3 ;
	$c = 7E-10 ;
		\end{lstlisting}
	\end{block}
\end{frame}

\begin{frame}[containsverbatim]{Chaîne de caractères}
	\begin{itemize}
		\item suite de caractères, de longueur 0 à 2Gio, codage unicode (UTF-8 ou UTF-16) en interne PHP, mais il faut transcoder explicitement ou automatiquement
		\item utilisation d'échappement des caractères : \texttt{\textbackslash{}n}, \texttt{\textbackslash{}r}, \texttt{\textbackslash{}t}, \texttt{\textbackslash{}v}, \texttt{\textbackslash{}\textbackslash{}}, \texttt{\textbackslash{}'}, \texttt{\textbackslash{}"}, \texttt{\textbackslash{}\$}, \ldots
		\item chaînes de caractères littérales : pas de substitution de variable possible
		\item chaînes de caractères interprétable : des substitutions de variables peuvent intervenir
		\item quatre représentations possibles : \texttt{''}, \texttt{""}, \texttt{nowdoc}, \texttt{heredoc}
	\end{itemize}
	\begin{block}{code : chaînes de caractères}
		\begin{lstlisting}
<?php
	$n = 27 ;
	echo 'une chaine de $n caracteres' ; // une chaine de $n caracteres
	echo "une chaine de $n caracteres" ; // une chaine de 27 caracteres
	
	$chaine = <<<"EOCHAINE"
Une chaine de caracteres NOWDOC sur plusieurs
lignes, de plus de $n caracteres.
EOCHAINE;
/* Une chaine de caracteres NOWDOC sur plusieurs
lignes, de plus de 27 caracteres. */

	$chaine = <<<EOCHAINE
Une chaine de caracteres HEREDOC sur plusieurs
lignes, de plus de $n caracteres.
EOCHAINE;
/* Une chaine de caracteres HEREDOC sur plusieurs
lignes, de plus de $n caracteres. */
		\end{lstlisting}
	\end{block}
\end{frame}

\begin{frame}[containsverbatim]{Booléen}
	\begin{itemize}
		\item \texttt{true} et \texttt{false}
		\item les autres types de valeurs peuvent être interprétés comme des booléens
	\end{itemize}
	\begin{block}{code : booléens}
		\begin{lstlisting}
<?php
	$a = true ;
	$b = false ;
	$c = $a && $b ; // false !
	$c == false ; // true !
	$c == "" ; // true !
	$c === "" ; // false !?
		\end{lstlisting}
	\end{block}
\end{frame}

\begin{frame}[containsverbatim]{Tableau}
	\begin{itemize}
		\item représente une liste ordonnée de valeurs
		\item les valeurs peuvent être de tout type PHP
		\item les valeurs sont indicées par un scalaire : entier, chaîne de caractère, flottant, \ldots
		\item les valeurs et les indices peuvent être de types différents dans un même tableau, et chaque élément peuvent accueillir successivement des valeurs de types différents
	\end{itemize}
	\begin{block}{code : tableaux}
		\begin{lstlisting}
<?php
	$a = array () ;
	$b = array (1, 2, $a) ;
	$c = array (1 => "toto", 'b' => "tata") ;
	$d = array (
		array (true => "vrai"),
		false => 2,
	) ;
	$e = [1, &$b, 3, "a", ["g" => 6, new MaClasse ()]] ;
	$e[4] = "b" ;
	$e[5]['g'] = null ;
	$e = [1, &$b, 3, "b", array ('g' => null, new MaClasse ())] ; // True !
		\end{lstlisting}
	\end{block}
\end{frame}

\begin{frame}[containsverbatim]{Objet}
	\begin{itemize}
		\item découle directement de l'instanciation d'une classe
		\item l'accès aux membres de la classe se fait via l'opérateur \texttt{->}
		\item sa valeur découle directement de ses propriétés 
	\end{itemize}
	\begin{block}{code : objets}
		\begin{lstlisting}
<?php
	class foo {
		function do_foo () {
			echo "Doing foo." ;
		}
	}

	$bar = new foo () ;
	$bar->do_foo () ;
		\end{lstlisting}
	\end{block}
\end{frame}

\begin{frame}[containsverbatim]{Sans type}
	\begin{itemize}
		\item \texttt{null} représente une absence de valeur
		\item lorsqu'on affecte \texttt{null} à une variable, elle est libérée et les références à la valeur remplacée aussi !
	\end{itemize}
	\begin{block}{code : null}
		\begin{lstlisting}
<?php
	$a = null ;
	isset ($a) == false ; // true !
	$b = 1 ;
	$c = & $b ;
	$b = null ;
	$c == null ; // true !
	array () == null ; // false !
	"" == null ; // false !
	false == null ; false !
	$m = new MaClasse () ;
	unset ($m) ;
	$m == null ; // true !
	$n = new MaClasse () ;
	$n = null ;
	$n == null ; // true, evidemment.
		\end{lstlisting}
	\end{block}
\end{frame}

\subsection{Variables}

\begin{frame}[containsverbatim]{Variable}
	\begin{itemize}
		\item représente une valeur nommée ; ou peut ne pas avoir de valeur (\texttt{null})
		\item est, implicitement (en interne), un pointeur sur une valeur
		\item \texttt{\$[\_a-zA-Z][\_a-zA-Z0-9]*}, des variables sont déjà définies par PHP, certaines peuvent être reécrites, d'autres non : \texttt{\$GLOBALS}, \texttt{\$\_SERVER}, \ldots
		\item n'est pas typée, peut contenir successivement plusieurs valeurs de types différents
		\item a un domaine d'existence : portée ou scope
	\end{itemize}
	\begin{block}{code : variables}
		\begin{lstlisting}
<?php
	$a = 1 ;
	$_MaVariable = "Ma Valeur" ;
	$Tableau = array ("toto", 5) ;
	$Objet = new user ('toto') ;
	$texte = $GLOBALS["Objet"]->say () ;
	$var = '_MaVariable' ;
	$test = $$var == array ("toto", 5) ; // True !
	function add ($a, $b) {
		return $a + $b ;
	}
		\end{lstlisting}
	\end{block}
\end{frame}

\begin{frame}[containsverbatim]{Référence}
	\begin{itemize}
		\item une référence peut être vue comme un pointeur, mais \textbf{ne} peut \textbf{pas} être manipulé en tant que tel (ainsi qu'on peut le faire avec des pointeurs en \texttt{C})
		\item toute variable, peut être vue comme un pointeur sur une valeur, certaines variables pointent sur la même instance de valeur (même zone mémoire), on les appelle des références
		\item une valeur n'est plus accessible dès lors que toutes les références à celle-ci sont détruites (via \texttt{unset ()} ou \texttt{= null})
		\item une référence à une valeur est créée avec l'opérateur \texttt{\&}
	\end{itemize}
	\begin{block}{code : références}
		\begin{lstlisting}
<?php
	$a = 1
	$b = & $a ;
	$b = 2 ;
	$a == 2 ; // True !
	unset ($a) ;
	$b == 2 ; // True !
	function & incremente (&$a, $b = 1) {
		$a += $b ;
		return $a ;
	}
		\end{lstlisting}
	\end{block}
\end{frame}

\subsection{Constantes}

\begin{frame}[containsverbatim]{Constantes}
	\begin{itemize}
		\item valeur nommée, mais inchangeable
		\item le nom de la constante suit les mêmes règles de nommage des variables, mais sans le \texttt{\$}
		\item habituellement, en PHP, les noms des constantes sont en majuscule
		\item des constantes magiques existent (des constantes qui changent de valeur :) ) : \texttt{\_\_LINE\_\_}, \texttt{\_\_FILE\_\_}, \texttt{\_\_DIR\_\_}, \texttt{\_\_FUNCTION\_\_}, \texttt{\_\_CLASS\_\_}, \ldots
	\end{itemize}
	\begin{block}{code : constantes}
		\begin{lstlisting}
<?php
	define ("MA_CONSTANTE", "sa valeur") ;
	echo MA_CONSTANTE ;
		\end{lstlisting}
	\end{block}
\end{frame}

\subsection{Commentaires et documentation}

\begin{frame}[containsverbatim]{Commentaires}
	\begin{itemize}
		\item destinés aux développeurs (et aux curieux)
		\item démystifient le code, explicite un code un peu confus ou complexe ou optimisé
		\item permettent la lecture rapide d'un code sans avoir à le décrypter
		\item doivent toujours être à jour vis-à-vis du code ! 
	\end{itemize}
	\begin{block}{code : commentaires}
		\begin{lstlisting}
<?php
	// Si $a est non null et que la valeur incrementee n'est pas zero...
	if (isset ($a) && $a ++) {
		$b += $a ; // ...on ajoute $a a $b
	}
	
	// ------------8<--------------
	// parfois des modifications qu'il faudra enlever avant la prod
	// ------------>8--------------	
	
	/* Un commentaire
	sur plusieurs lignes si necessaire. */
		\end{lstlisting}
	\end{block}
\end{frame}

\begin{frame}[containsverbatim]{Documentation}
	\begin{itemize}
		\item destinée aux développeurs et utilisateurs du code
		\item renseigne sur le comportement et la composition des objets définis (fichiers, classes, méthodes, fonctions, variables, constantes, \ldots)
		\item doit être systématique, exhaustive, à jour, structurée !
		\item doxygen, javadoc, \ldots
	\end{itemize}
	\begin{block}{code : documentation}
		\begin{lstlisting}
<?php
	/** Classe representant les donnees personnelles d'un utilisateur. */
	class user {
	
		/** nom de l'utilisateur : chaine de 32 caracteres UTF-8 */
		private $name ;
		
		/** Constructeur de la classe
		  * Initialise le nom de l'utilisateur.
		  * \param $name : une chaine de caracteres, tronquee a 32 caracteres, 'toto' si non renseigne ou null.
		  */
		function __construct ($name = 'toto') {
			$this->name = substr (0, 32, $name) ;
		}
		
		/** L'utilisateur se presente.
		  * \return une chaine de caractere presentant l'utilisateur.
		  */
		function say () {
			return "Bonjour, je m'appelle {$this->name}." ;
		}
	}
		\end{lstlisting}
	\end{block}
\end{frame}

\subsection{Expressions}

\begin{frame}[containsverbatim]{Expression}
	Combinaison (via des opérateurs ou des fonctions) de valeurs qui retourne une valeur
	\begin{itemize}
		\item \lstinline!1!, \lstinline!array ('nom' => 'toto')!, \lstinline!$t['nom']! : une valeur ; \lstinline!$a! : une valeur de variable
		\item \lstinline!$a + sqrt (2)!, \lstinline!$a <= 5!, \lstinline!$a ++! : une expression mathématique, booléenne
		\item \lstinline!strlen ('toto' . "titi $a")! : un appel de fonction
		\item \lstinline!isset ($a) ? "" : 3! : une opération quelconque
		\item \lstinline!(1 + strlen ('toto')) * 2)! : gestion des priorités avec \texttt{()}
		\item \lstinline!$a = 1! : est une expression, elle retourne la valeur de \lstinline!$a! après affectation
		\item n'importe quelle combinaison d'expressions
		\item le type résultant de l'expression est déterminé lors de son évaluation (typage faible)
	\end{itemize}
	Ne sont pas considérées comme expressions :
	\begin{itemize}
		\item \lstinline!if ($a == null) { /* ... */ }!, \lstinline!while (true) { /* ... */ }!
		\item \lstinline*echo "hello world!"* : \texttt{echo} n'est pas une fonction, mais une structure du langage PHP
		
	\end{itemize}
\end{frame}

\begin{frame}[containsverbatim]{Opérateurs}
	\begin{itemize}
		\item La précédence des opérateurs : \texttt{()} (\url{http://fr2.php.net/manual/en/language.operators.precedence.php})
		\item Les opérateurs arithmétiques : \lstinline!-$a!, \lstinline!$a + $b!, \lstinline!$a - $b!, \lstinline!$a * $b!, \lstinline!$a / $b!, \lstinline!$a % $b!, \lstinline!$a ** $b! 
		\item Les opérateurs d'affectation : \lstinline!$a = 0!, \lstinline!$a += 1!, \lstinline!$a -= 1!, \lstinline!$a *= 1!, \lstinline!$a /= 1!, \lstinline!$a .= 'toto'!, \ldots
		\item Opérateurs sur les bits : \lstinline!~$a!, \lstinline!$a & $b!, \lstinline!$a | $b!, \lstinline!$a ^ $b!, \lstinline!$a << $b!, \lstinline!$a >> $b!
		\item Opérateurs de comparaison : \lstinline!$a == $b!, \lstinline!$a === $b!, \lstinline*$a != $b*, \lstinline!$a <> $b!, \lstinline!$a < $b!, \lstinline!$a > $b!, \lstinline!$a <= $b!, \lstinline!$a >= $b!, \lstinline*$a !== $b*, 
		\item Opérateur de contrôle d'erreur : \lstinline!@ $a / 0! (ignore les erreurs générées par l'expression, bof :/)
		\item Opérateur d'exécution : \lstinline!`ls -lA`! (exécute une commande système et retourne la sortie standard)
		\item Opérateurs d'incrémentation et décrémentation : \lstinline!$a ++!, \lstinline!$a --!, \lstinline!-- $a!, \lstinline!++ $a!
		\item Les opérateurs logiques : \lstinline!$a and $b!, \lstinline!$a or $b!, \lstinline!$a xor $b!, \lstinline*!$a*, \lstinline!$a && $b!, \lstinline!$a || $b! 
		\item Opérateurs de chaînes : \lstinline!$a . $b!, \lstinline!$a .= $b!
		\item Opérateurs de tableaux : \lstinline!$a + $b!, \lstinline!$a == $b!, \lstinline!$a === $b!, \lstinline*$a != $b*, \lstinline!$a <> $b!, \lstinline*$a !== $b*
		\item Opérateurs de types : \lstinline!$a instanceof 'Classe'!, \lstinline!$a instanceof Classe!
	\end{itemize}
\end{frame}

\subsection{Instructions}

\begin{frame}[containsverbatim]{Instruction}
	\begin{itemize}
		\item Affectation ou expression
		\item finit par \texttt{;} (ou \texttt{?>})
		\item en général, une par ligne
		\item une instruction retourne une valeur (la plupart du temps)
	\end{itemize}
	\begin{block}{code : instructions}
		\begin{lstlisting}
<?php
	$a = 1 ;
	$a ++ ;
	$c = new MaClasse () ;
	printf ("%s", $c) ;
	return 'toto' ;
	'toto' ; // pas tres utile
	$a = $b = (int) 3 + strlen ((string) $c) ;
	$a = $b == $c ;
	echo $a ;
		\end{lstlisting}
	\end{block}
\end{frame}

\subsection{Contrôle de l’exécution}

\begin{frame}[containsverbatim]{Exécution conditionnelle} %if
	\begin{block}{code : if}
		\begin{lstlisting}
<?php
	if (/* expression */) {
		/* instructions */
	}
	
	if (/* expression */) {
		/* instructions */
	} else {
		/* instructions */
	}
	
	if (/* expression */) {
		/* instructions */
	} elseif { // ou else if
		/* instructions */
	}
		\end{lstlisting}
	\end{block}
\end{frame}

\begin{frame}[containsverbatim]{Exécution en boucle} %while
	\begin{itemize}
		\item boucle conditionnelle (tant que) : \texttt{while () \{\}}
		\item boucle sur liste ou tableau (pour chaque élément) : \texttt{foreach (as) \{\}}
		\item boucle itérative : \texttt{for (;;) \{\}}
		\item contrôle des boucles via : \lstinline!break ;!, \lstinline!continue ;!
	\end{itemize}
	\begin{block}{code : for, foreach, while}
		\begin{lstlisting}
<?php
	while (/* expression */) {
		/* instructions */
	}
	
	$tableau = array (/* .. */) ;
	foreach ($tableau as $element) {
		/* instruction */
	}
	
	foreach ($tableau as $index => $element) {
		/* instructions */
	}
	
	for ($i = 0 ; $i < 10 ; $i ++) {
		/* instructions */
	}
		\end{lstlisting}
	\end{block}
\end{frame}

\begin{frame}[containsverbatim]{Exécution sélective} %switch
	\begin{itemize}
		\item exécution conditionnelle selon plusieurs valeurs scalaires
	\end{itemize}
	\begin{block}{code}
		\begin{lstlisting}
<?php
	$n = (int) random (10) ;
	switch ($n) {
	case 1 :
		echo 'unite' ;
		break ;
	case 2 :
	case 4 :
		echo 'pair inferieur a 5' ;
		break ;
	case 3 :
		echo 'impair inferieur a 5' ;
		break ;
	default :
		echo 'superieur ou egal a 5' ;
	}
		\end{lstlisting}
	\end{block}
\end{frame}

\begin{frame}[containsverbatim]{Exceptions} % try
	\begin{itemize}
		\item permet de capturer une erreur d'exécution, de traiter l'erreur et finir le boulot
		\item les exceptions retournées par PHP sont des objets de classes dérivant de \texttt{Exception}
	\end{itemize}
	\begin{block}{code}
		\begin{lstlisting}
<?php
	function diviser ($a, $b) {
		try {
			if ($b == 0) { 
				throw new Exception ("Division par 0.") ;
			}
			return $a / $b ;
		} catch (Exception $e) {
			return null ;
		}
	}
		\end{lstlisting}
	\end{block}
\end{frame}

\subsection{Fonctions}

\begin{frame}[containsverbatim]{Fonctions} %function name()
	\begin{itemize}
		\item déclaration : signature + corps (sauf pour les méthodes abstraites)
		\item l'identificateur (nom) : \texttt{[\_a-zA-Z][\_a-zA-Z0-9]*}
		\item requiert zéro ou plusieurs paramètres selon la signature
		\item le corps est compris entre \{ et \}, démarre par la première instruction et finit par la dernière instruction ou par l'instruction \lstinline!return /* ... */ ;!
		\item retourne une seule valeur (via \texttt{return}) ou une référence (défini dans la signature par \&) vers la valeur
		\item a un domaine d'existence : portée, scope
	\end{itemize}
	\begin{block}{déclaration}
		\begin{lstlisting}
<?php
function ajouter ($a, $b = 0) {
	return $a + $b ;
}

function & incrementer (&$a, $b = 1) {
	$a += $b ;
	return $a ;
}
		\end{lstlisting}
	\end{block} 
\end{frame}

\begin{frame}[containsverbatim]{Paramètres de fonctions}
	Ou arguments.
	\begin{itemize}
		\item passage par valeur : \lstinline!function afficher ($a) { echo $a ; }!, une copie de la valeur est effectuée au moment de l'appel de la fonction, la valeur \texttt{\$a} dans la fonction est indépendante de la valeur du code appelant
		\item passage par référence : \lstinline!function incrementer (& $a) { $a ++ ; }!, une référence de la valeur est passée à la fonction, toute modification de \texttt{\$a} dans la fonction est effectuée sur la valeur de la variable dans le code appelant
		\item un paramètre peut être optionnel, mais il lui faut une valeur par défaut : \lstinline!function ajouter ($a, $b = 1) { return $a + $b ; }!, lors de l'appel de cette fonction, on peut omettre le paramètre (\lstinline!$c = ajouter (10, 1) ; $d = ajouter (2) ;!)
		\item un paramètre peut être typé (PHP $\ge$ 5.1) par \texttt{array}, \texttt{objet}, \texttt{callable}, mais pas par les types scalaires !
		\item lors de l'appel, on peut préciser plus de paramètres que déclarés dans la signature : liste variable de paramètres. Dans la fonction, on les récupère en tant que tableau de valeurs (ou de références) via les fonctions \texttt{func\_get\_args ()}, \texttt{func\_num\_args ()}
		\item depuis PHP 5.6, on peut utiliser le mot clé \texttt{...} avant le paramètre dans la signature pour préciser que tous les paramètres suivants sont stockés dans ce paramètre : \lstinline!function ajouter (...$num) { $s = 0 ; foreach ($num as $n) { $s += $n ; } return $s ; }! 
	\end{itemize}
\end{frame}

\begin{frame}[containsverbatim]{Fonctions anonymes} %function ()
	\begin{itemize}
		\item fonction à usage unique ou jetable, une fonction anonyme n'a pas de nom : \lstinline!function ($a) { /* corps */ }!
		\item s'utilise dans certains contextes
	\end{itemize}
	\begin{block}{code : fonctions anonymes}
		\begin{lstlisting}
<?php
$ajouter = function ($a, $b) { return $a + $b ; }
$s = $ajouter (1, 2) ;

$a = array (/* ... */) ;
$b = array_walk (function ($e) { return $e + 1 ; }, $a) ;
		\end{lstlisting}
	\end{block}
\end{frame}


\subsection{Classes}

\begin{frame}[containsverbatim]{Les classes}
	PHP $\ge$ 5.0.0 devient orienté objet.
	\begin{itemize}
		\item notion de classes classique : classe, objet/instance, abstraite, interface, héritage (simple), surcharge, \ldots
		\item \texttt{traits} \ldots
		\item définition et implémentation dans le même fichier (différent de java, par exemple)
	\end{itemize}
	\begin{block}{définition de classe}
		\begin{lstlisting}
// definition
class A {

	private $var ;

	function afficher () { /* ... */ } 
	/* ... */
} 

// heritage
class B extends A { /* ... */ } 

// definition d'interface
interface iC { /* ... */ } 

// definition classe abstraite et heritage d'interface
abstract class D extends iC { /* ... */ } 
		\end{lstlisting}
	\end{block}
\end{frame}

\begin{frame}[containsverbatim]{Les classes : \guillemotleft~normales~\guillemotright}
	\begin{block}{définition de classe}
		\begin{lstlisting}
class A {

	/** Ma donnee : un entier */
	private $var ;

	/** Constructeur
	  * Initialise ma donnee a 0.
	  */
	function __construct () {
		$this->var = 0 ;
	} 
	
	/** Affiche ma donnee avec un suffixe.
	  * \param texte : suffixe
	  */
	public function afficher ($texte) {
		echo $this->var, " ", $texte ;
	}
	
	/** Ajoute 1 a ma donnee.
	  */
	public function incrementer () {
		$this->var ++ ;
	} 
} 
		\end{lstlisting}
	\end{block}
\end{frame}

\begin{frame}[containsverbatim]{Les classes : les méthodes}
	\begin{itemize}
		\item \lstinline!public function nom ($parametre) { /* corps */ }! ou \lstinline!function nom ($parametre) { /* corps */ }! : méthode publique, utilisable par tout le monde, redéfinissable dans les classes descendantes
		\item \lstinline!private function nom ($param1, $param2) { /* corps */}! : méthode privée, utilisable depuis un objet de la même classe
		\item \lstinline!protected function nom ($param1, $param2) { /* corps */}! : méthode protégée, utilisable depuis un objet de la même classe ou descendante, redéfinissable
		\item \lstinline!static function nom ($param) { /* corps */}! : méthode statique ou méthode de classe, utilisable sans instancier la classe
		\item \lstinline!final function nom ($param) { /* corps */}! : méthode finale, non redéfinissable
		\item \lstinline!abstract function nom ($param) ;! : méthode abstraite, sans corps, obligatoirement redéfinie dans la classe descendante, implique que la classe où elle est déclarée est aussi abstraite.
		\item \lstinline!public function __get ($name) { /* corps */ }! : méthode magique (ici un getter, surcharge de l'opération de lecture d'une propriété), \ldots
	\end{itemize}
\end{frame}

\begin{frame}[containsverbatim]{Les classes : les propriétés}
	Attributs, membres, champs, \ldots
	\begin{itemize}
		\item \lstinline!public $var ;! : déclaration d'une propriété publique (accessible par tout le monde), sans initialisation
		\item \lstinline!public $var = 'valeur' ;! : déclaration avec une valeur initiale
		\item \lstinline!private $var = array (10, 'val') ;! : propriété privée, la valeur initiale est un tableau de valeurs
		\item \lstinline!protected $var = 6 ;! : propriété protégée
		\item \lstinline!static $var = 2 ;! : propriété statique, propriété de classe, accessible lorsque la classe n'est pas instanciée
	\end{itemize}
\end{frame}

\begin{frame}[containsverbatim]{Les classes : instanciation, libération}
	Instanciation : 
	\begin{itemize}
		\item \lstinline!$o = new MyClass ('val') ;!
		\item crée une instance de la classe MyClass, appelle la méthode \lstinline!MyClass::__construct ($var)! si elle est définie
	\end{itemize}
	Libération :
	\begin{itemize}
		\item \lstinline!unset $o ;!
		\item libère l'objet \lstinline!$o! après avoir appelé la méthode \lstinline!MyClass::__destruct ()! si elle est définie
		\item ne libère pas automatiquement les propriétés de type objet ou référencés, à faire explicitement dans \lstinline!__destruct ()!
	\end{itemize}
\end{frame}

\begin{frame}[containsverbatim]{Les classes : visibilité}
	Concerne les méthodes et les propriétés, statiques ou pas.
	\begin{itemize}
		\item \texttt{public} : publique, pas de restriction de visibilité
		\item \texttt{protected} : protégée, visibilité et utilisabilité restreinte à la classe, ses descendants et ascendants
		\item \texttt{private} : privée, visibilité et utilisabilité restreinte à la classe où c'est déclaré 
		\item sans mot clé \texttt{public}, \texttt{protected}, \texttt{private}, la visibilité par défaut est publique.
		\item \lstinline!var $var ;! est identique à \lstinline!public $var ;!
		\item Lors de l'héritage, les classes descendantes peuvent restreindre la visibilité, pas l'étendre : \texttt{public} \textrightarrow \texttt{protected}, \texttt{public} \textrightarrow \texttt{private}, \texttt{protected} \textrightarrow \texttt{private}, \sout{\texttt{public} \textleftarrow \texttt{private}}, \ldots
	\end{itemize}
\end{frame}

\begin{frame}[containsverbatim]{Les classes : redéfinition et surcharge}
	Redéfinition de méthode :
	\begin{itemize}
		\item changement de comportement de la méthode de même nom dans la classe fille
		\item on ne redéfinit que les méthodes accessibles depuis la classe fille, la visibilité de la redéfinition est la même ou restreinte
		\item l'ancêtre de la méthode redéfinie reste accessible en la nommant explicitement : \lstinline!ClasseParent::MaMethode () ;! ou \lstinline!parent::MaMethode () ;!
	\end{itemize}
	Surcharge de méthode :
	\begin{itemize}
		\item rajoute du comportement à la méthode de même nom dans la classe fille
		\begin{itemize}
			\item appelle dans le corps de la méthode \lstinline!MaMethode()! la méthode ancêtre via \lstinline!parent::MaMethode()!
			\item se fait notamment pour \lstinline!__construct ()! et \lstinline!__destruct ()!
		\end{itemize}
		\item rajoute un comportement à une opération sur la classe ou l'objet : méthode magique
		\begin{itemize}
			\item \lstinline!__get ($prop)! : est appelée lors de l'accès à la propriété \lstinline!$prop! inaccessible (restreinte ou inexistante)
			\item \lstinline!__set ($prop, $value)! : est appelée lors de l'affectation d'une valeur à une propriété inaccessible
			\item \lstinline!__call ($func, $params)! : est appelée lors de l'appel d'une méthode inaccessible
			\item \lstinline!__isset ()! : est appelée lors du test \lstinline!isset ($o)!
			\item \ldots
		\end{itemize}
	\end{itemize}
\end{frame}

\begin{frame}[containsverbatim]{Les classes : héritage}
	\begin{itemize}
		\item héritage classique, non multiple
		\item implémentation d'interface multiples
		\item la classe mère doit être évidemment définie et chargée avant la classe fille
	\end{itemize}
	\begin{block}{code : héritage}
		\begin{lstlisting}
<?php
abstract class MaClasse { /* ... */ }

class MaClasseFille extends MaClasse { /* ... */ }

interface iIFace1 { /* ... */ }

interface iIFace2 { /* ... */ }

class Classe extends MaClasse implements iIFace1, iIFace2 { /* ... */ } 
		\end{lstlisting}
	\end{block}
\end{frame}

\subsection{Fichiers}

\begin{frame}[containsverbatim]{Scripts}
	\begin{itemize}
		\item \texttt{script} : ensemble du code entre l'appel et la fin d'exécution
		\item extension habituelles : \texttt{.php} (script chapeau), \texttt{.php.inc} (fichiers inclus), \texttt{.ini} (configuration), \texttt{.php.class} (définition de classe), \ldots
		\item le code PHP dans un fichier doit être précédé de \texttt{<?php}
		\item le code PHP dans un fichier doit être suivi de \texttt{?>} ou pas !
		\item on fait référence à un fichier PHP via son chemin dans le système de fichier (et pas l'URL) relativement au chemin du script chapeau ou de façon absolue.
	\end{itemize}
	\begin{block}{fichier.php}
		\begin{lstlisting}
<?php
	echo 'Salut le monde !' ;
		\end{lstlisting}
	\end{block}
\end{frame}

\begin{frame}[containsverbatim]{Inclusions}
	\begin{itemize}
		\item fragmente les gros blocs de code monolithique
		\item regroupe les fonctionnalités
		\item 4 types d'inclusions :
		\begin{itemize}
			\item inclusion simple : \lstinline$include ("fichier.php.inc") ;$ (continue le script si le fichier est manquant, peut être inclus plusieurs fois)
			\item inclusion unique : \lstinline$include_once ("fichier.php.inc") ;$ (continue le script si le fichier est manquant, n'est pas inclus si le fichier l'a déjà été durant le script)
			\item inclusion obligatoire simple : \lstinline$require ("fichier.php.inc") ;$ (le script s'arrête si le fichier est manquant, peut être inclus plusieurs fois)
			\item inclusion obligatoire unique : \lstinline$require_once ("fichier.php.inc") ;$ (le script s'arrête si le fichier est manquant, n'est pas inclus si le fichier l'a déjà été durant le script)
		\end{itemize}
	\end{itemize}
	\begin{block}{inclusions}
		\begin{lstlisting}
<?php
	include ("repertoire/fichier.php.inc") ;
	include_once ("autre_fichier.php.inc") ;
	require ("../autrerepertoire/fichier3.php.inc") ;
	require_once ("/home/user1/fichier4.php.inc") ;
		\end{lstlisting}
	\end{block}
\end{frame}


\subsection{Espaces de nom}

\begin{frame}[containsverbatim]{Espaces de nom, définition}
	Depuis PHP $\ge$ 5.3.0
	\begin{itemize}
		\item encapsulation/nommage d'objets (classes, interface, fonctions, constantes)
		\item éviter les collisions
		\item créer des alias
	\end{itemize}
	\begin{block}{fichier1.php (définition)}
		\begin{lstlisting}
<?php
	namespace MonProjet ;
	
	const CONNEXION_OK = 1 ;
	class Connexion { /* ... */ }
	function connecte() { /* ... */  }
?>
		\end{lstlisting}
	\end{block}
	\begin{block}{fichier2.php (définition sous-espace)}
		\begin{lstlisting}
<?php
	namespace MonProjet\SousEspace ;

	function disconnect () { /* ... */ }
?>
		\end{lstlisting}
	\end{block}
\end{frame}

\begin{frame}[containsverbatim]{Espaces de nom, utilisation}
	\begin{itemize}
		\item nom non qualifié : \texttt{Foo()}
		\item nom qualifié : \texttt{SousEspace\textbackslash{}Foo()}
		\item nom absolu : \texttt{\textbackslash{}MonProjet\textbackslash{}SousEspace\textbackslash{}Foo()}
	\end{itemize}
	\begin{block}{importation et utilisation}
		\begin{lstlisting}
<?php
	use MonProjet ;
	
	echo CONNEXION_OK ;
	$a = new Connexion() ;
	$b = connecte() ;
	SousEspace\disconnect() ;
?>
		\end{lstlisting}
	\end{block}
\end{frame}


\subsection{Application}

\begin{frame}{Application PHP}
	\begin{itemize}
		\item pas de notion d'application
		\begin{itemize}
			\item pas d'objet syntaxique particulier
			\item pas de limitation de fichier (\texttt{chroot})
			\item deux applications PHP peuvent interférer
		\end{itemize}
		\item une application PHP est et devrait être définie par le développeur
		\begin{itemize}
			\item configuration (une par application, par exemple)
			\item variables globales
			\item éventuellement un espace de nom global
			\item circonscrite dans un espace de fichier (sécurité : \texttt{chroot}, \texttt{apache}, \ldots)
			\item utilisation d'un script \emph{chapeau}
		\end{itemize}
	\end{itemize}
\end{frame}



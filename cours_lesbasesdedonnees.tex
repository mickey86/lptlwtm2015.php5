% Utilisation du langage — Les bases de données

\section{Les bases de données}

\begin{frame}[containsverbatim]{Généralités}
	\begin{itemize}
		\item Permet de gérer d’une façon structurée et efficace (du point de vue intégrité, sécurité) des données souvent volumineuses
		\item PHP permet d’utiliser plusieurs types de bases de données : les relationnelles et autres \guillemotleft~NoSQL~\guillemotright ; parmi les SGBDR supportés : MySQL (MariaDB), PostgreSQL, Oracle, Informix, DB2, SQLite, etc\ldots ; chaque SGBDR ayant des spécificités, mais aussi un langage commun.
		\item PHP permet d’utiliser chacun de ces bases via un driver : spécifique à la base, ou générique 
		\begin{itemize}
			\item \textbf{PDO} (PHP Data Objects) : l’API actuellement fournie avec PHP d’accès générique aux bases de données ; nécessite un driver PDO d’accès spécifique
			\item ODBC : nécessite une couche d’accès (côté système)
			\item DBA (Data Base Abstraction layer) : obsolète, PDO la remplace
		\end{itemize}
		\item Habituellement : 
		\begin{enumerate}
			\item configuration et ouverture d’une connexion vers la base de données choisie
			\item envoi d’une ou plusieurs requêtes via la connexion ouverte et réception des données
			\item traitement des données (répéter 2 et 3 autant de fois que nécessaire)
			\item fermeture de la connexion
		\end{enumerate}
	\end{itemize}
\end{frame}
 
\begin{frame}[containsverbatim]{Fonctions : PDO}
	\begin{block}{Accès à une base, requête et affichage du résultat}
		\begin{lstlisting}
$dsn = 'mysql:dbname=testdb;host=127.0.0.1' ;
$user = 'dbuser' ;
$pass = 'dbpass' ;

try {
    $dbh = new PDO ($dsn, $user, $password) ;
} catch (PDOException $e) {
    echo 'Echec connexion : ' . $e->getMessage () ;
}

$query = 'SELECT * FROM users AS U WHERE U.active = :active' ;

$stmt = $dbh->prepare ($query) ;
$results = $stmt->execute (array ("active" => 1)) ;

foreach ($results->fetchAll () as $r) {
	print_r ($r) ;
}
unset ($dbh) ;
		\end{lstlisting}
	\end{block}
	Habituellement, on crée un objet dérivant de \texttt{PDO} pour stocker la configuration d’accès à la base,  gérer la connexion et la déconnexion automatique, et éventuellement les transactions\ldots \\
	\url{http://fr2.php.net/manual/fr/book.pdo.php}
\end{frame}
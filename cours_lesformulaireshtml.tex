% Utilisation du langage — Les formulaires HTML

\section{Les formulaires HTML}

\begin{frame}[containsverbatim]{Généralités 1/2}
	\begin{itemize}
		\item Il s’agit d’une manière d’interagir avec l’utilisateur : l’application lui met à disposition un ensemble de champs avec des valeurs par défaut ou non, et l’utilisateur décide de les remplir ou pas, décide de renvoyer le formulaire ou pas.
		\begin{alertblock}{Pire !}
			L’utilisateur peut (avec des outils faciles à se procurer) renvoyer n’importe quelle réponse au serveur : mal remplir les champs, les remplir avec du code injecté, supprimer ou rajouter des champs (même s’ils n’existent pas dans le formulaire), envoyer trop de données, \ldots
		\end{alertblock}
		\begin{block}{Bonne pratique}
			Vérifier et valider systèmatiquement les données renvoyées par le navigateur.
		\end{block}
	\end{itemize}
\end{frame}

\begin{frame}[containsverbatim]{Généralités 2/2}
	\begin{itemize}
		\item Le protocole HTTP offre 4 méthodes d’interaction du client avec le serveur :
		\begin{itemize}
			\item \texttt{GET} : demande une \emph{ressource} via une URL qui peut embarquer des paramètres (\textit{query string} \textit{url-encoded})
			\item \texttt{POST} : envoit un ensemble de données (\textit{payload} \textit{url-encoded}) représentant une \emph{ressource} à un script qui est aussi une ressource se trouvant à l’URL précisée (dans l’attribut HTML \texttt{action} de la balise \texttt{form})
			\item \texttt{PUT} : envoit un ensemble de données (\textit{url-encoded}) représentant une nouvelle \emph{ressource} à un script qui est une ressource se trouvant à l’URL précisée en vue d’un ajout de données
			\item \texttt{DELETE} : demande la suppression d’une \emph{ressource} précisée par son URL (qui peut contenir des paramètres \textit{url-encoded})
			\item en retour de chacune de ces méthodes, le serveur est censé retourner une ressource (page HTML, ou un autre document, ou un fichier, etc\ldots)
		\end{itemize}
		\item En pratique, on n’utilise que les méthodes \texttt{GET} et \texttt{POST} (les autres méthodes restant exploitables)
		\item Le navigateur et PHP s’arrangent à coder et décoder les données ; ainsi les données reçues et envoyées par PHP ou le navigateur sont transmises sans perte à l’autre partie
	\end{itemize}
\end{frame}

\begin{frame}[containsverbatim]{Données de formulaire 1/2}
	\begin{block}{formulaire HTML}
		\begin{lstlisting}[language=html]
<form action="foo.php" method="post" enctype="multipart/form-data" >
    <label for="inputname">Name</label>
	<input id="inputname" type="text" name="username" /><br />
    <label for="inputemail">Email</label>
	<input id="inputemail" type="text" name="email" /><br />
	<label for="inputsexmale">male</label>
	<input id="inputsexmale" type="radio" name="sex" value="male" />
	<label for="inputsexfemale">female</label>
	<input id="inputsexfemale" type="radio" name="sex" value="female" />
	<label for="inputtopicscooking">cooking</label>
	<input id="inputtopicscooking" type="checkbox" name="topics[]" value="cooking" />
	<label for="inputtopicsgardening">gardening</label>
	<input id="inputtopicsgardening" type="checkbox" name="topics[]" value="gardening" />
	<label for="inputcolors">Color</label>
	<select id="inputcolor" name="color">
		<option value="none" checked="checked">none of these</option>
		<option value="red">red</option>
		<option value="green">green</option>
		<option value="blue">blue</option>
	</select>
	<label for="inputcomment">Comment</label>
	<textarea id="inputcomment" name="comment"></textarea>
	<label for="inputavatar">Avatar</label>
	<input id="hiddenmaxfilesize" type="hidden" name="MAX_FILE_SIZE" value="30000" />
	<input id="inputavatar" type="file" name="avatar" />
    <input type="inputsubmit" name="submit" value="submit" />
    <input type="inputcancel" name="cancel" value="cancel" />
</form>
		\end{lstlisting}
	\end{block}
\end{frame}

\begin{frame}[containsverbatim]{Données de formulaire 2/2}
	\begin{block}{données côté PHP}
		\begin{lstlisting}
$_POST == array (
	"username" => "mikael",
	"email" => "mikael.cordon@atos.net",
	"sex" => "male",
	"topics" => array ("cooking", "gardening"),
	"color" => "red",
	"comment" => "I love PHP!",
	"MAX_FILE_SIZE" => 30000,
	"submit" => "submit",
) ; // True
$_FILES == array (
	"avatar" => array (
		"name" => "mikael_avatar.gif",
		"type" => "image/gif",
		"size" => "3004",
		"tmp_name" => "YDjgznris",
		"error" => 0,
	),
) ;
if (false === ($utilisateur = UtilisateurDAO.rechercherParNom($_POST["username"]))) {
	// Insertion !
	if (false === move_uploaded_file($_FILES["avatar"]["tmp_name"], 
			"/srv/share/img/" . $_FILES["avatar"]["name"])) {
		// Erreur copie sur le serveur !
	}
} else {
	// Erreur $utilisateur existe deja !
}
		\end{lstlisting}
	\end{block}
\end{frame}

% Utilisation du langage — Les tableaux
% Les fonctions array*

\section{Manipulation des tableaux}

\begin{frame}{Généralités}
	\begin{itemize}
		\item Un objet du type \texttt{array} permet de stocker des paires clé-valeurs de façon ordonnée.
		\item Les clés sont des valeurs scalaires, les valeurs sont de tous types (même des tableaux)
		\item Ainsi les tableaux avec les fonctions adaptées permettent de gérer des structures de données telles que : tableau, liste, file, pile, table de hachage, dictionnaire, arbre, ensemble, tas, etc\ldots
		\item La limite en taille de la structure est celle imposée par l’environnement (système et configuration PHP)
		\item Une des structures les plus utilisées de PHP.
	\end{itemize}
\end{frame}

\begin{frame}[containsverbatim]{Fonctions}
	Constructions :
	\begin{block}{construction de tableaux}
		\begin{lstlisting}
<?php
$tableau = array ("un", "deux", 3) ;
$tableau2 = array ("one" => "un", "two" => "deux", "three" => 3) ;
$tableau3 = array (array ("toto" => "titi"), "tutu") ;
$tableau4["tata"] = $tableau3[0] ;
		\end{lstlisting}
	\end{block}
	\begin{itemize}
		\item Expression (\texttt{+}) : \lstinline~$tableau = $t1 + array ("valeur") ;~
		\item Conversion : \lstinline~explode()~, \lstinline~(array)~
		\item Génération : \lstinline~array_diff()~, \lstinline~array_fill()~, \lstinline~array_merge()~, \lstinline~array_slice()~, etc\ldots
	\end{itemize}
	Navigation:
	\begin{itemize}
		\item \lstinline~reset()~, \lstinline~prev()~, \lstinline~next()~, \lstinline~current()~, \lstinline~end()~ 
		\item \lstinline~$tableau["tata"]~, \lstinline~$tableau["tata"][3]~
	\end{itemize}
	Tri, filtre, modification :
	\begin{itemize}
		\item \lstinline~array_filter()~, \lstinline~sort()~, \lstinline~ksort()~, \lstinline~array_walk()~, \lstinline~array_splice()~, etc\ldots
		\item \lstinline~array_pop()~, \lstinline~array_push()~, \lstinline~array_shift()~, \lstinline~array_unshift()~, \lstinline~array_unique()~, etc\ldots
	\end{itemize}
		Tests et calculs :
	\begin{itemize}
		\item \lstinline~count()~, \lstinline~in_array()~, \lstinline~array_key_exists()~, etc\ldots
		\item \lstinline~array_reduce()~, \lstinline~array_search()~, etc\ldots
	\end{itemize}
	\url{http://fr2.php.net/manual/en/ref.array.php}\\
\end{frame}
 

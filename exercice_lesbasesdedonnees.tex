% Exercices sur la base de données.

\section{Exercices : Accéder à une base de données en lecture et écriture}

\begin{frame}[containsverbatim]{Manipulation \textnumero{1} : Générer une base de données}
	Via une interface d’administration de \texttt{MySQL} (\texttt{PHPMyAdmin}, \texttt{mysql} en console ou autres clients avec droit d’administration) :
	\begin{enumerate}
		\item Se connecter à l’interface
		\item Créer une nouvelle base de données \texttt{projetexercice}
		\item Créer un utilisateur ayant accès à la nouvelle base depuis la machine PHP
		\item Charger la structure de la base de données depuis le script \texttt{projet\_createdb.sql}
		\item Examiner la structure, la comprendre.
	\end{enumerate}
\end{frame}

\begin{frame}[containsverbatim]{Exercice \textnumero{2} : Accès à la base depuis PHP}
	Programmer dans un nouveau projet (répertoire dédié) :
	\begin{enumerate}
		\item Créer \texttt{./index.php} et \texttt{./Base.php.class}
		\item \texttt{Base.php.class} contient la définition d’une classe dérivant de PDO dont le constructeur fixe les paramètres de la connexion à la base.
		\item dans \texttt{index.php} on instancie un objet de classe \texttt{Base} ; puis on effectue la requête suivante :
			\begin{block}{Requête SQL}
				\begin{lstlisting}
SELECT U.*
FROM utilisateur AS U
ORDER BY U.login ASC
			\end{lstlisting}
		\end{block}
		\item  Afficher le résulat.
	\end{enumerate}
\end{frame}
 
\begin{frame}[containsverbatim]{Exercice \textnumero{3} : Manipulation d’objets}
	Programmer dans le projet de l’exercice \textnumero{2} :
	\begin{enumerate}
		\item Créer le fichier \texttt{Utilisateur.php.class} contenant une classe représentant l’entité correspondante dans la base (une propriété par champ, de nom et type correspondant)
		\item Créer le fichier \texttt{UtilisateurDAO.php.class} contenant une classe statique implémentant les opérations de base d’accès aux entités de l’interface \texttt{UtilisateurDAOIface.php.class}.
		\item Tester chaque opération :
		\begin{itemize}
			\item Créer une instance de l’entité \texttt{Utilisateur}
			\item L’insérer dans la base
			\item Changer une information de l’utilisateur (pas sa clé primaire) et mettre à jour dans la base
			\item La récupérer de la base
			\item L’afficher
			\item La supprimer de la base
		\end{itemize}
	\end{enumerate}
\end{frame}
 

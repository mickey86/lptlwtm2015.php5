% Exercices sur les chaînes de caractères et les tableaux.

\section{Exercices : les chaînes et les tableaux}

\begin{frame}[containsverbatim]{Exercice \textnumero{1} : manipulation de chaînes}
	Dans un répertoire dédié (à cette série d’exercices), rapatrier le fichier \texttt{lorem\_ipsum.txt}.
	Pour charger le contenu de ce fichier dans une variable, utiliser le code suivant :
	\begin{block}{chargement d’une chaîne de caractères à partir d’un fichier}
		\begin{lstlisting}
<?php
define ("LOREM_FILE", "./lorem_ipsum.txt") ;

// Si le fichier existe on charge son contenu dans la variable ; sinon, fin du programme.
if (file_exists (LOREM_FILE)) {
	$texte_orig = file_get_contents (LOREM_FILE) ;
} else {
	die ("Le fichier " . LOREM_FILE . " n'est pas disponible.") ;
}
		\end{lstlisting}
	\end{block}
	Programmer :
	\begin{enumerate}
		\item Afficher le contenu de la variable \texttt{\$texte\_orig}
		\item Changer le codage caractère de \texttt{ISO8859-15} à \texttt{UTF-8} et afficher le résultat
		\item Remplacer dans le texte les caractères accentués par leur homologue sans accent et afficher le résultat.
		\item Justifier le texte à 100 caractères (chaque ligne ne doit pas contenir plus de 100 caractères) et afficher le résultat.
		\Pitem Justifier sans couper les mots.
	\end{enumerate}
\end{frame}

\begin{frame}[containsverbatim]{Exercice \textnumero{2} : manipulation de tableaux}
	En reprenant le code de l’exercice précédent\ldots
	Programmer :
	\begin{enumerate}
		\item Insérer chaque ligne du texte justifié comme élément dans un tableau, et afficher le résultat.
		\item Créer un tableau de statistiques répertoriant pour chaque ligne : nombre de mots, nombre de lettre, et nombre d’occurrence de chaque lettre.
		\item Regrouper les statistiques des occurrences des lettres dans un nouveau tableau.
		\item Trier le tableau des statistiques globales, par lettre, l’afficher ; puis par nombre d’occurrences et l’afficher.
	\end{enumerate}
\end{frame}

% Cours Les règles d’écriture

\section{Règles d’écriture}

\begin{frame}{Pourquoi ?}
	\begin{itemize}
		\item Fait partie des bonnes pratiques de développement (quelque soit le langage)
		\item Facilité de lecture/compréhension du code (pour les auteurs et les développeurs futurs)
		\item Systématiser/Automatiser les recherches/traitements du code (notamment lors de l’utilisation de dépôts de code qui analyse les différences entre versions)
		\item Mettre en place des métriques du code (contrôle qualité)
	\end{itemize}
\end{frame}

\begin{frame}{Noms}
	Nommage des éléments :
	\begin{itemize}
		\item Les noms des éléments (variables, fonctions, constantes, classes, etc…)
		\item \textbf{camelBack}, première lettre de chaque élément du nom, sauf premier élément :
		\begin{itemize}
			\item classes : \lstinline&class cPerson \{ /* .. */ \}&, \lstinline&interface iPerson \{ /* .. */ \}&
			\item variables : \lstinline&\$person = new cPerson () ; &, \lstinline&\$person->lastName = "Dupond" ; &, \lstinline&\$person->firstName = "Jean" ;& 
			\item fonctions : \lstinline&function hasChildren () \{ /* .. */ \}&, \lstinline&\$person->sayItsName () ;&
		\end{itemize}
		\item Avec des \textbf{underscores}, les éléments sont séparés par un underscore :
		\begin{itemize}
			\item classes : \lstinline&class person_class \{ /* .. */ \}&, \lstinline&interface person_iface \{ /* .. */ \}&
			\item variables : \lstinline&\$person = new person_class () ; &, \lstinline&\$person->last_name = "Dupond" ; &, \lstinline&\$person->first_name = "Jean" ;& 
			\item fonctions : \lstinline&function has_children () \{ /* .. */ \}&, \lstinline&\$person->say_its_name () ;&
		\end{itemize}
		\item \ldots
		\item choisir un schéma, une convention de nommage et s’y tenir !
	\end{itemize}
	But :
	\begin{itemize}
		\item Déterminer rapidement à la lecture du nom ce que peut contenir une variable ou ce que fait une fonction
		\item Avec une convention bien établie et bien réfléchie : pouvoir deviner efficacement des noms de fonctions ou des noms de classes, etc…
		\item Construire, invoquer ou rechercher de manière systématique le nom des éléments
		\item Fait partie des bonnes pratiques de programmation (pas qu’en PHP)
	\end{itemize}
\end{frame}

\begin{frame}[containsverbatim]{Espaces}
	La mise en page du code : 
	\begin{itemize}
		\item espacement horizontal : début de ligne (indentation), entre les éléments du langage, à la fin de la ligne
		\item espacement vertical : entre les instructions, entre les blocs de code et les macros-éléments (classes, fonctions)
		\item placement des accolades, des parenthèses et indentation : regroupement et hiérarchisation du code
	\end{itemize}
	But : 
	\begin{itemize}
		\item Déterminer au premier coup d’œil l’architecture du code (les blocs et la hiérarchie) : où débute et où termine le corps des boucles, fonctions, classes, etc…
		\item Regrouper/Scinder les opérations unitaires, les groupes fonctionnels…
	\end{itemize}
\end{frame}

\begin{frame}[containsverbatim]{Mise en page — mauvaise}
	\begin{block}{pas mis en page}
		\begin{lstlisting}
<?php $t = array("ceci est une chaine de caracteres"   ,1024,
false,"une autre chaine");
foreach ($t as $i => $s) {
echo "La ".$i+1."e valeur ";
// string ?


if(is_string($s)) {echo "est une chaine de caracteres.";} 
else
	echo "n'est pas une chaine de caracteres.";echo "\n"     ;}
		\end{lstlisting}
	\end{block}
\end{frame}

\begin{frame}[containsverbatim]{Mise en page — bonne}
	\begin{block}{bien mis en page}
		\begin{lstlisting}
<?php
$mesValeurs = array (
	"ceci est une chaine de caracteres", 
	1024,
	false,
	"une autre chaine",
) ;

// On boucle sur les elements de mesValeurs
foreach ($mesValeurs as $indice => $valeur) {
	$numero = $indice + 1 ;
	echo "La {$numero}e valeur " ;

	// On determine si la valeur est de type string
	if (is_string ($valeur)) {
		echo "est une chaine de caracteres." ;
	} else {
		echo "n'est pas une chaine de caracteres." ;
	}
	
	echo "\n" ;
}
		\end{lstlisting}
	\end{block}
\end{frame}

\begin{frame}{Découpage du code}
	\begin{itemize}
		\item Garder le principe \textbf{KISS} (\emph{Keep It Simple, Stupid}) : une fonction, une classe, un fichier, une variable doit répondre à une problématique/fonctionnalité simple.
		\item Découper les gros problèmes en autant de sous-problèmes simples que nécessaire : diviser pour mieux régner.
	\end{itemize}
	But : 
	\begin{itemize}
		\item Avoir le moins de code possible à lire pour trouver ce qu’on cherche.
		\item Réutiliser des éléments atomiques est plus efficace que les macro-éléments
		\item Fichiers : chargement (depuis le disque + analyse) plus efficace si les fichiers sont petits
	\end{itemize}
	Exemples :
	\begin{itemize}
		\item Les parties du modèle applicatif (MVC, PO/BO/DAO) : chacun un répertoire.
		\item Un module (sous-partie d’application) : un répertoire.
		\item Une définition de classe : un fichier.
		\item Une entité : une classe.
		\item Une fonction : une opération atomique technique, ou fonctionnelle.
		\item Une donnée : une variable.
		\item Une variable : un seul type de donnée.
		\item \ldots
	\end{itemize}
\end{frame}


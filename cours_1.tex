% Cours №1

\section{Cours \textnumero{}1 : présentation de PHP}

\subsection{Présentation}

\begin{frame}{Historique}
    \begin{itemize}
	    \item 1994 : Rasmus Lerdorf, \og~Personal Home Page / Forms Interpreter~\fg
    	\item \og~PHP : Hypertext Preprocessor~\fg 
    	\item PHP~4 (2000) : version très utilisée (autrefois), \textbf{n’est pas} orientée objet
    	\item PHP~5 (2004) : version de PHP \textbf{orientée objet}
    	\item version actuelle (2014-08) : 5.6
    \end{itemize}
\end{frame}

\begin{frame}{Qu'est-ce que PHP ?}
    \begin{block}{\url{http://php.net/}}PHP is a popular general-purpose scripting language that is especially suited to web development.\\Fast, flexible and pragmatic, PHP powers everything from your blog to the most popular websites in the world.
    \end{block}
    \begin{itemize}
        \item Logiciel libre : PHP license v3.01 (BSD-style) ; Documentation : CC-By PHP Documentation Group v3
        \item langage à script : interprétation à l’exécution
        \item langage de haut niveau : accès fichiers, sessions, caches, protocole HTTP, les codages caractères (UTF-8)\ldots
        \item beaucoup de bibliothèques disponibles : PEAR, PECL (extensions)
    \end{itemize}
    Documentation de référence : \url{http://www.php.net/manual}
\end{frame}

\begin{frame}{Technologies associées}
	Rarement utilisé seul, PHP s’associe, via des bibliothèques et/ou nativement, à :
	\begin{itemize}
		\item httpd : apache, nginx, lighttpd, IIS, \ldots
		\item bases de données : MySQL/MariaDB, PostgreSQL, Oracle, DB2, \ldots
		\item présentation : html (html-4.0, XHTML, XML, HTML5), css, javascript, templates, \ldots
		\item communications : fichiers, flux, http/s, ftp/s, \ldots
		\item \ldots
	\end{itemize}
\end{frame}

\subsection{Architecture}

\begin{frame}{Fonctionnement WEB : client/serveur}
	\begin{itemize}
		\item[1] client : le navigateur WEB (Firefox, Chromium, Lynx, IE, \ldots)
			\begin{itemize}
				\item effectue des requêtes : URL (requête HTTP + commande GET, POST, PUT, DELETE)
				\item interprète la réponse reçue : HTML, fichier, CSS, javascript, erreurs (404, 500, 501, \ldots) 
			\end{itemize} 
		\item[2] serveur : httpd 
			\begin{itemize}
				\item httpd ouvre, décode éventuellement la session HTTPS
				\item httpd interprète les URL et appelle le script PHP
				\item PHP exécute le script et génère une réponse HTTP : header HTTP + document
			\end{itemize}
		\item protocole de communication : HTTP 1.1 (éventuellement encapsulé dans un tunnel SSL)
		 	\begin{itemize}
		 		\item transporte la requête HTTP (URL + entêtes)
		 		\item transporte la réponse HTTP (entêtes + document)
		 	\end{itemize}
	\end{itemize}
\end{frame}

\begin{frame}{Installation}
	\begin{itemize}
		\item PHP est un exécutable ; les permissions et droits de l'utilisateur exécutant et ceux sur les fichiers et l'exécutable s'appliquent :
			\begin{itemize}
				\item l'utilisateur et l'environnement de httpd en mode WEB
				\item l'utilisateur courant et son environnement en mode CLI
			\end{itemize}
		\item PHP est livré avec des bibliothèques et des fichiers de configuration
	\end{itemize}
	Installation :
	\begin{itemize}
		\item GNU/Linux : paquet \texttt{php} directement depuis le dépôt (\texttt{.deb}, \texttt{.rpm}, \texttt{.tgz}, \ldots) ; penser aux paquets \texttt{libapachemod-php5}, \ldots
		\item Microsoft Windows : \texttt{WAMP} (Windows Apache MySQL PHP)
		\item Apple MACOSX : \texttt{MAMP} (Mac Apache MySQL PHP), \texttt{AMPPS}, \ldots
	\end{itemize}
\end{frame}

\begin{frame}{Configuration}
	Plusieurs notions :
	\begin{itemize}
		\item environnement : prod, dev, test, \ldots
		\item fonctionnalités : dés/activer des extensions, type et niveau d'erreurs, \ldots
		\item paramétrage : accès mail, accès bdd, \ldots
		\item sécurité : exposition de données sensibles, niveaux de chiffrages, hachage, \ldots
	\end{itemize}
	Plusieurs niveaux :
	\begin{itemize}
		\item global : \texttt{php.ini} (\texttt{/etc/php/})
		\item utilisateur : \texttt{.user.ini} (\texttt{\textasciitilde/}, \texttt{\$HOME/})
		\item service : \texttt{php.ini} (\texttt{/etc/apache/conf.d/})
		\item projet : configuration httpd (\texttt{apache.conf}, \texttt{.htaccess})
		\item \emph{on-the-fly} : \texttt{ini\_get()} et \texttt{ini\_set()}
	\end{itemize}
\end{frame}

\begin{frame}{Quelques éléments de configuration}
	Dans \texttt{php.ini} :
	\begin{itemize}
		\item \texttt{max\_execution\_time=30}, \texttt{memory\_limit=128M} : limiter les ressources utilisées par chaque instance de script
		\item \texttt{post\_max\_size=8M}, \texttt{upload\_max\_filesize=2M} : limiter la taille des données
		\item \texttt{session.use\_cookies=1} : paramètres d'extension (ici, \texttt{session})
		\item \ldots
	\end{itemize}
\end{frame}
